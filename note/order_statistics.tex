\documentclass[addpoints,12pt,a4paper]{exam}
\usepackage{etoolbox}
\newbool{ans}
\booltrue{ans}
\ifbool{ans}{\printanswers}{}

\setlength\fillinlinelength{2cm}
\setlength\answerclearance{2ex}

\usepackage[AutoFakeBold,AutoFakeSlant]{xeCJK}
\usepackage{CJKnumb}
\setCJKmainfont[AutoFakeSlant=.1,AutoFakeBold=2]{cwTeX Q Kai Medium} 
\setCJKfamilyfont{kaiv}[Vertical=RotatedGlyphs]{cwTeX Q Medium}

\usepackage{amsmath,amsfonts,amssymb,amsthm,graphicx,hyperref,booktabs,tabularx}

\DeclareMathOperator\dom{dom}
\DeclareMathOperator\ran{ran}
\DeclareMathOperator\var{var}
\newcommand{\ds}{\displaystyle}
\newcommand{\prb}{\mathsf{P}}
\newcommand{\expc}{\mathsf{E}}

\usepackage{enumitem}
\pagestyle{headandfoot}
\firstpageheadrule
%\firstpageheader{}{1121\enskip\enskip 數\ \ 學\ \ 一\enskip\enskip BMA10101\enskip\enskip 微\ \ 積\ \ 分\ \ 第\ \ 十\ \ 次\ \ 隨\ \ 堂\ \ 考\ \ 試\ \ 題}{}
\firstpagefooter{}{第\thepage\ 頁(共\numpages 頁)}{}
\runningfooter{}{第\thepage\ 頁(共\numpages 頁)}{}
\footrule
\extraheadheight{-3mm}
\extrafootheight{-8mm}
\extrawidth{3cm}

\setenumerate{label=(\alph*)}%,itemsep=1pt,topsep=3pt}

%\renewcommand{\solutiontitle}{
%  \noindent\textbf{解:  }
%}

%% no box for solutions
%\unframedsolutions
%\setlength\linefillheight{.5in}
\newcommand{\ie}{\,\Longrightarrow\,}
\newcommand{\ifff}{\,\Longleftrightarrow\,}

\usepackage{pgfplots}

% colorblind-friendly palette
% mixed colours: CB sees contrasting grays
\definecolor{M1}{RGB}{0,0,0}
\definecolor{M2}{RGB}{0,73,73}
\definecolor{M3}{RGB}{0,146,146}
\definecolor{M4}{RGB}{255,109,182}
\definecolor{M5}{RGB}{255,182,119}
% cool colours: CB sees contrasting blues
\definecolor{C1}{RGB}{73,0,146}
\definecolor{C2}{RGB}{0,109,219}
\definecolor{C3}{RGB}{182,109,255}
\definecolor{C4}{RGB}{109,182,255}
\definecolor{C5}{RGB}{182,219,255}
% warm colours: CB sees contrasting yellow
\definecolor{W1}{RGB}{146,0,0}
\definecolor{W2}{RGB}{146,73,0}
\definecolor{W3}{RGB}{219,209,0}
\definecolor{W4}{RGB}{36,255,36}
\definecolor{W5}{RGB}{255,255,109}

%%%%%%%%%%%%%%%%%%%%%%%%%%%%%%%%%%%%%%%%%%%%%%%%%%%%%%%%%%%%%%%%%%%%%

\usepackage{multicol}

\begin{document}

%\begin{center}
%  學號:\fillin[][5cm] $\qquad$ 姓名:\fillin[][5cm]
%\end{center}

\begin{questions}
  \pointname{\%}

  \question Let $X$ be a random variable with probability density function $\ds f(x) = \begin{cases}2x & \text{for } 0 < x < 1 \\ 0 & \text{otherwise} \end{cases}$. \\A sample of size $3$ is randomly selected from this distribution. Let $Y$ be a random variable representing the median value from the sample; calculate the variance of $Y$.
  \begin{solution} (Ross S. M. ``A First Course in Probability'', Tenth Edition, Pearson. Section 6.6 Order Statistics, c.f. \href{https://www.soa.org/4ae412/globalassets/assets/files/edu/2024/2024-01-exam-p-syllabus.pdf}{Exam P syllabus}) \\
    Let $X_1$, $X_2$, $\ldots$, $X_n$ be $n$ i.i.d. continuous r.v. having a common density function $f$ and distribution $F$. Define
  \begin{align*}
    X_{(1)} &= \text{smallest of }X_1,\,X_2,\,\ldots,\,X_n \\
    X_{(2)} &= \text{second smallest of }X_1,\,X_2,\,\ldots,\,X_n \\
    \vdots  & \\
    X_{(j)} &= j\text{-th smallest of }X_1,\,X_2,\,\ldots,\,X_n \\
    \vdots  & \\
    X_{(n)} &= \text{largest of }X_1,\,X_2,\,\ldots,\,X_n \\
  \end{align*}
  The ordered values $X_{(1)}\leqslant X_{(2)}\leqslant \cdots\leqslant X_{(n)}$ are known as the \emph{order statistics} corresponding to the r.v.s $X_1,\,X_2,\,\ldots,\,X_n$. The joint pdf of the order statistics is obtained by noting that $X_{(1)},\,X_{(2)},\,X_{(n)}$ will take on the values $x_1\leqslant x_2\leqslant\cdots\leqslant x_n$ iff for some permutation $(i_1,\,i_2,\,\ldots,\,i_n)$ of $(1,\,2,\,\ldots,\,n)$, $X_1 = x_{i_1},\, X_2 = x_{i_2},\,\ldots,\,X_n = x_{i_n}$. Since for any permutation $(i_1,\,i_2,\,\ldots,\,i_n)$ of $(1,\,2,\,\ldots,\,n)$, 
  \begin{align*}
    &\prb\Big\{x_{i_1} - \frac{\varepsilon}{2} < X_1 < x_{i_1} + \frac{\varepsilon}{2},\;x_{i_2} - \frac{\varepsilon}{2} < X_2 < x_{i_2} + \frac{\varepsilon}{2},\;\ldots,\;x_{i_n} - \frac{\varepsilon}{2} < X_n < x_{i_n} + \frac{\varepsilon}{2}\Big\} \\
  \approx\;&\;\varepsilon^n f_{X_1,\,X_2,\,\ldots,\,X_n}(x_{i_1},\,x_{i_2},\,\ldots,\,x_{i_n}) \\
  =\;&\;\varepsilon^n f(x_{i_1})\cdot f(x_{i_2})\cdots f(x_{i_n}) \\
  =\;&\;\varepsilon^n f(x_{1})\cdot f(x_{2})\cdots f(x_{n}) 
  \end{align*}
  It follows that, for $x_1 < x_2 < \ldots < x_n$, 
  \begin{align*}
    &\prb\Big\{x_{1} - \frac{\varepsilon}{2} < X_{(1)} < x_{1} + \frac{\varepsilon}{2},\;x_{2} - \frac{\varepsilon}{2} < X_{(2)} < x_{2} + \frac{\varepsilon}{2},\;\ldots,\;x_{n} - \frac{\varepsilon}{2} < X_{(n)} < x_{n} + \frac{\varepsilon}{2}\Big\} \\
    \approx\;&\;n!\,\varepsilon^n f(x_{1})\cdot f(x_{2})\cdots f(x_{n}) 
  \end{align*}
  Dividing by $\varepsilon^n$ and letting $\varepsilon\to 0$ yields
  \begin{align*}
    f_{X_{(1)},\,X_{(2)},\,\ldots,\,X_{(n)}}(x_{1},\,x_{2},\,\ldots,\,x_{n}) = n!\,f(x_1)\cdot f(x_2)\cdots f(x_n),\quad x_1 < x_2 < \ldots < x_n
  \end{align*}
  The density function of $X_{(j)}$ can be obtained as follows: in order for $X_{(j)}$ to equal $x$, it is necessary for $j - 1$ of the $n$ values $X_1,\,X_2,\,\ldots,\,X_n$ to be less than $x$, $n - j$ of them to be greater than $x$, and exactly $1$ of them to equal $x$; the probability of this group is $\ds F(x)^{j - 1}(1 - F(x))^{n - j} f(x)$, and there are $\ds\frac{n!}{(j - 1)!(n - j)!}$ such groups, so the probability density function of $X_{(j)}$ is given by
  \begin{align*}
    f_{X_{(j)}}(x) = \frac{n!}{(n - j)!\,(j - 1)!}\,F(x)^{j - 1}(1 - F(x))^{n - j}f(x)
  \end{align*}

  Now $n = 3$, $j = 2$, so $f(y)$, the pdf of $Y$, is simply $\ds\frac{3!}{(3 - 2)!\,(2 - 1)!}\,(y^2)^{2 - 1}(1 - y^2)^{3 - 2}\,2y = 6\,y^2\cdot(1 - y^2)\cdot 2y = 12(y^3 - y^5)$, $\ds\expc\{Y\} = \int_0^1 y\cdot f(y)\,\text{d}y = 12\int_0^1 y^4 - y^6\,\text{d}y = 12\,\Big(\frac{1}{5} - \frac{1}{7}\Big) = \frac{24}{35}$, $\ds\expc\{Y^2\} = \int_0^1 y^2\cdot f(y)\,\text{d}y = 12\int_0^1 y^5 - y^7\,\text{d}y = 12\,\Big(\frac{1}{6} - \frac{1}{8}\Big) = \frac{1}{2}$, so $\ds\var{Y} = \expc\{Y^2\} - (\expc\{Y\})^2 = \frac{1}{2} - \frac{24^2}{35^2} = 0.029795918$.
  \end{solution}

\end{questions}
\end{document}
