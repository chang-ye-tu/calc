\documentclass[12pt]{article}
\usepackage[margin=15mm]{geometry}
\usepackage{amsmath,amssymb,amsthm}
\usepackage{mathtools}
\usepackage{physics}
\usepackage{enumitem}
\usepackage{tcolorbox}
\usepackage{booktabs}
\usepackage{hyperref}
\usepackage{parskip}

% 中文支援
\usepackage{xeCJK}
\setCJKmainfont{Noto Serif CJK TC}
\setCJKsansfont{Noto Sans CJK TC}
\setCJKmonofont{Noto Sans Mono CJK TC}

\tcbuselibrary{theorems, skins, breakable}

\newtcolorbox{application}[1][]{
  colback=blue!5!white,
  colframe=blue!75!black,
  fonttitle=\bfseries,
  title={應用 --- #1},
  breakable
}

\newtcolorbox{keyintegral}[1][]{
  colback=green!5!white,
  colframe=green!50!black,
  fonttitle=\bfseries,
  title={#1},
  breakable,
  after skip=1.5em
}

\newtcolorbox{duis}{
  colback=orange!5!white,
  colframe=orange!75!black,
  fonttitle=\bfseries,
  title={Leibniz 積分法則 --- 積分號下取微分},
  breakable
}

\newtcolorbox{solution}{
  colback=gray!5!white,
  colframe=gray!75!black,
  fonttitle=\bfseries,
  title={解答},
  breakable
}

\newtheorem{theorem}{定理}[section]
\newtheorem{lemma}[theorem]{引理}
\newtheorem{proposition}[theorem]{命題}
\theoremstyle{definition}
\newtheorem{definition}[theorem]{定義}
\newtheorem{example}[theorem]{例題}
\theoremstyle{remark}
\newtheorem*{remark}{註}
\newtheorem*{motivation}{物理動機}

\newcommand{\R}{\mathbb{R}}
\newcommand{\N}{\mathbb{N}}
\newcommand{\E}{\mathbb{E}}

\setCJKmainfont{cwTeX Q Yuan Medium}
\newcommand{\ds}{\displaystyle}
\newcommand{\ie}{\;\Longrightarrow\;}
\newcommand{\ifff}{\;\Longleftrightarrow\;}
\newcommand{\orr}{\;\vee\;}
\newcommand{\andd}{\;\wedge\;}
\newcommand{\mi}{\mathrm{i}}
\newcommand{\llt}{\left\langle}
\newcommand{\rgt}{\right\rangle}
\DeclareMathOperator*{\dom}{dom}
\DeclareMathOperator*{\codom}{codom}
\DeclareMathOperator*{\ran}{ran}
\DeclareMathOperator*{\sgn}{sgn}
\DeclareMathOperator*{\degr}{deg}
\newcommand{\floor}[1]{\lfloor #1 \rfloor}
\newcommand{\ceil}[1]{\lceil #1 \rceil}
\newcommand{\proj}[2]{\mathrm{proj}_{\,#2}\,#1}

\theoremstyle{definition}
\newtheorem*{dfn}{定義}
\newtheorem*{prp}{性質}
\newtheorem*{fact}{結論}
\newtheorem*{thm}{定理}
\newtheorem*{ex}{例}
\newtheorem*{sol}{解}
\newtheorem*{prf}{證}
\newtheorem*{rmk}{註}
\newtheorem*{exe}{習題}

\title{\texorpdfstring{\vspace{-16mm} 積分技巧:積分號下取微分法}{積分技巧:積分號下取微分法}} 
\author{\vspace{-5em}}
\date{\vspace{-4em}}

\begin{document}
\maketitle

\begin{duis}
若 $f(x, \alpha)$ 連續且偏導數 $\frac{\partial f}{\partial \alpha}$ 連續,則在適當條件下:
\[
\frac{\mathrm{d}}{\mathrm{d}\alpha} \int_a^b f(x, \alpha)\, \mathrm{d}x = \int_a^b \frac{\partial f}{\partial \alpha}(x, \alpha)\, \mathrm{d}x
\]
更一般地,若積分上下限也依賴於 $\alpha$:
\[
\frac{\mathrm{d}}{\mathrm{d}\alpha} \int_{a(\alpha)}^{b(\alpha)} f(x,\alpha)\,\mathrm{d}x = f(b(\alpha),\alpha)\cdot b'(\alpha) - f(a(\alpha),\alpha)\cdot a'(\alpha) + \int_{a(\alpha)}^{b(\alpha)} \frac{\partial f}{\partial \alpha}\,\mathrm{d}x
\]
\end{duis}

%\textbf{策略:}給定困難積分 $I$,引入參數 $\alpha$ 建立 $I(\alpha)$。若 $I'(\alpha)$ 較容易計算,我們可以透過對 $\alpha$ 積分來還原 $I$。

%=============================================================================
\section{敘述與證明}
%=============================================================================
\subsection{有限區間的 Leibniz 積分法則}

\begin{theorem}[Leibniz 積分法則]\label{thm:leibniz-finite}
設 $f(x, \alpha)$ 定義於 $[a,b] \times [\alpha_0, \alpha_1]$。若
\begin{enumerate}[label=(\roman*)]
  \item $f(x, \alpha)$ 在 $[a,b] \times [\alpha_0, \alpha_1]$ 連續,
  \item $\frac{\partial f}{\partial \alpha}(x, \alpha)$ 存在且在 $[a,b] \times [\alpha_0, \alpha_1]$ 連續,
\end{enumerate}
則
\[
\frac{\mathrm{d}}{\mathrm{d}\alpha} \int_a^b f(x, \alpha)\, \mathrm{d}x = \int_a^b \frac{\partial f}{\partial \alpha}(x, \alpha)\, \mathrm{d}x.
\]
\end{theorem}

\begin{prf}
定義 $F(\alpha) = \int_a^b f(x, \alpha)\,\mathrm{d}x$。我們需證明
\[
\lim_{h \to 0} \frac{F(\alpha + h) - F(\alpha)}{h} = \int_a^b \frac{\partial f}{\partial \alpha}(x, \alpha)\,\mathrm{d}x
\]

\textbf{步驟 1:建立差商。}
\[
\frac{F(\alpha + h) - F(\alpha)}{h} = \frac{1}{h}\int_a^b \bigl[f(x, \alpha+h) - f(x,\alpha)\bigr]\,\mathrm{d}x = \int_a^b \frac{f(x, \alpha+h) - f(x,\alpha)}{h}\,\mathrm{d}x
\]

\textbf{步驟 2:應用均值定理。}

對於每個固定的 $x$,函數 $\alpha \mapsto f(x, \alpha)$ 可微。由均值定理,存在 $\theta_x \in (0, 1)$(依賴於 $x$ 和 $h$)使得
\[
\frac{f(x, \alpha + h) - f(x, \alpha)}{h} = \frac{\partial f}{\partial \alpha}(x, \alpha + \theta_x h)
\]
因此
\[
\frac{F(\alpha + h) - F(\alpha)}{h} = \int_a^b \frac{\partial f}{\partial \alpha}(x, \alpha + \theta_x h)\,\mathrm{d}x
\]

\textbf{步驟 3:利用均勻連續性。}

由於 $\frac{\partial f}{\partial \alpha}$ 在緊集 $[a,b] \times [\alpha_0, \alpha_1]$ 上連續,它是\emph{均勻連續}的。即對於任意 $\varepsilon > 0$,存在 $\delta > 0$ 使得對所有 $x \in [a,b]$ 及所有 $\alpha', \alpha'' \in [\alpha_0, \alpha_1]$,
\[
|\alpha' - \alpha''| < \delta \implies \left|\frac{\partial f}{\partial \alpha}(x, \alpha') - \frac{\partial f}{\partial \alpha}(x, \alpha'')\right| < \varepsilon
\]

\textbf{步驟 4:估計誤差。}

對於 $|h| < \delta$,我們有 $|\theta_x h| \leqslant|h| < \delta$,故
\[
\left|\frac{\partial f}{\partial \alpha}(x, \alpha + \theta_x h) - \frac{\partial f}{\partial \alpha}(x, \alpha)\right| < \varepsilon \quad \text{對所有 } x \in [a,b]
\]
因此
\begin{align*}
\left|\frac{F(\alpha+h) - F(\alpha)}{h} - \int_a^b \frac{\partial f}{\partial \alpha}(x,\alpha)\,\mathrm{d}x\right| &= \left|\int_a^b \left[\frac{\partial f}{\partial \alpha}(x, \alpha + \theta_x h) - \frac{\partial f}{\partial \alpha}(x, \alpha)\right]\mathrm{d}x\right| \\
&\leqslant \int_a^b \left|\frac{\partial f}{\partial \alpha}(x, \alpha + \theta_x h) - \frac{\partial f}{\partial \alpha}(x, \alpha)\right|\mathrm{d}x \\
&< \int_a^b \varepsilon\,\mathrm{d}x = \varepsilon(b-a)
\end{align*}

\textbf{步驟 5:結論。}

由於 $\varepsilon > 0$ 是任意的,我們有
\[
\lim_{h \to 0} \frac{F(\alpha+h) - F(\alpha)}{h} = \int_a^b \frac{\partial f}{\partial \alpha}(x, \alpha)\,\mathrm{d}x
\]
因此 $F$ 可微且 $F'(\alpha) = \int_a^b \frac{\partial f}{\partial \alpha}(x,\alpha)\,\mathrm{d}x$。
\end{prf}

\subsection{透過控制收斂定理處理瑕積分的 Leibniz 法則}

對於瑕積分,我們需要更強的條件。最簡潔的方法是使用 Lebesgue 控制收斂定理。

\begin{theorem}[控制收斂定理]\label{thm:DCT}
設 $(X, \mu)$ 為測度空間,$\{f_n\}$ 為可測函數序列,滿足
\begin{enumerate}[label=(\roman*)]
  \item 當 $n \to \infty$ 時,$f_n(x) \to f(x)$ 逐點收斂
  \item 存在可積函數 $g$ 使得對所有 $n$ 及幾乎所有 $x$,$|f_n(x)| \leqslant g(x)$
\end{enumerate}
則 $f$ 可積且
\[
\lim_{n \to \infty} \int_X f_n\,\mathrm{d}\mu = \int_X f\,\mathrm{d}\mu
\]
\end{theorem}

\begin{theorem}[瑕積分下的微分]\label{thm:leibniz-improper}
  設 $f(x, \alpha)$ 定義於 $[a, \infty) \times (\alpha_0, \alpha_1)$。假設
  \begin{enumerate}[label=(\roman*)]
    \item 對每個 $\alpha \in (\alpha_0, \alpha_1)$,函數 $x \mapsto f(x, \alpha)$ 在 $[a, \infty)$ 上可積
    \item 對每個 $x \in [a, \infty)$,偏導數 $\frac{\partial f}{\partial \alpha}(x, \alpha)$ 對所有 $\alpha \in (\alpha_0, \alpha_1)$ 存在
    \item 存在可積函數 $g: [a, \infty) \to [0, \infty)$ 使得
      \[
        \left|\frac{\partial f}{\partial \alpha}(x, \alpha)\right| \leqslant g(x) \quad \text{對所有 } x \in [a, \infty) \text{ 及 } \alpha \in (\alpha_0, \alpha_1)
      \]
  \end{enumerate}
  則
  \[
    \frac{\mathrm{d}}{\mathrm{d}\alpha} \int_a^{\infty} f(x, \alpha)\,\mathrm{d}x = \int_a^{\infty} \frac{\partial f}{\partial \alpha}(x, \alpha)\,\mathrm{d}x\]
\end{theorem}

\begin{prf}
  定義 $F(\alpha) = \int_a^{\infty} f(x, \alpha)\,\mathrm{d}x$。

\textbf{步驟 1:建立差商。}

對於充分小的 $h \neq 0$(使得 $\alpha + h \in (\alpha_0, \alpha_1)$)
\[
\frac{F(\alpha + h) - F(\alpha)}{h} = \int_a^{\infty} \frac{f(x, \alpha + h) - f(x, \alpha)}{h}\,\mathrm{d}x
\]

\textbf{步驟 2:應用均值定理。}

對於每個固定的 $x$ 對 $\alpha \mapsto f(x, \alpha)$ 應用均值定理,存在 $\theta_x \in (0, 1)$ 使得
\[
\frac{f(x, \alpha + h) - f(x, \alpha)}{h} = \frac{\partial f}{\partial \alpha}(x, \alpha + \theta_x h)
\]
定義
\[
\phi_h(x) = \frac{f(x, \alpha + h) - f(x, \alpha)}{h}
\]

\textbf{步驟 3:建立逐點收斂。}

當 $h \to 0$ 時
\[
\phi_h(x) = \frac{f(x, \alpha + h) - f(x, \alpha)}{h} \to \frac{\partial f}{\partial \alpha}(x, \alpha) \quad \text{對每個 } x
\]
這由偏導數的定義而得。

\textbf{步驟 4:找出控制函數。}

由均值定理的結果及條件 (iii)
\[
|\phi_h(x)| = \left|\frac{\partial f}{\partial \alpha}(x, \alpha + \theta_x h)\right| \leqslant g(x)
\]
對所有 $x \in [a, \infty)$ 及所有充分小的 $h$ 成立。

\textbf{步驟 5:應用控制收斂定理。}

我們有
\begin{itemize}
  \item 當 $h \to 0$ 時,$\phi_h(x) \to \frac{\partial f}{\partial \alpha}(x, \alpha)$ 逐點收斂
  \item $|\phi_h(x)| \leqslant g(x)$ 其中 $\int_a^{\infty} g(x)\,\mathrm{d}x < \infty$
\end{itemize}
由控制收斂定理
\[
\lim_{h \to 0} \int_a^{\infty} \phi_h(x)\,\mathrm{d}x = \int_a^{\infty} \lim_{h \to 0} \phi_h(x)\,\mathrm{d}x = \int_a^{\infty} \frac{\partial f}{\partial \alpha}(x, \alpha)\,\mathrm{d}x
\]

\textbf{步驟 6:結論。}

\[
F'(\alpha) = \lim_{h \to 0} \frac{F(\alpha + h) - F(\alpha)}{h} = \lim_{h \to 0} \int_a^{\infty} \phi_h(x)\,\mathrm{d}x = \int_a^{\infty} \frac{\partial f}{\partial \alpha}(x, \alpha)\,\mathrm{d}x
\]
\end{prf}

%\begin{example}
%考慮 $I(\alpha) = \int_0^{\infty} e^{-\alpha x} \frac{\sin x}{x}\,\mathrm{d}x$,其中 $\alpha > 0$。
%
%我們對 $\alpha \in [\alpha_0, \infty)$($\alpha_0 > 0$)驗證條件:
%\begin{enumerate}[label=(\roman*)]
%  \item 對每個 $\alpha > 0$,$\int_0^{\infty} e^{-\alpha x} \frac{\sin x}{x}\,\mathrm{d}x$ 收斂(被積函數在 $0$ 附近有界且指數衰減)。
%  \item $\frac{\partial}{\partial \alpha}\left(e^{-\alpha x}\frac{\sin x}{x}\right) = -e^{-\alpha x}\sin x$ 對所有 $x > 0$ 及 $\alpha > 0$ 存在。
%  \item 對於 $\alpha \geqslant \alpha_0$
%    \[\left|-e^{-\alpha x}\sin x\right| \leqslant e^{-\alpha_0 x}\] 
%    且 $\int_0^{\infty} e^{-\alpha_0 x}\,\mathrm{d}x = \frac{1}{\alpha_0} < \infty$。
%\end{enumerate}
%
%因此,積分號下微分是合理的:
%\[
%I'(\alpha) = -\int_0^{\infty} e^{-\alpha x}\sin x\,\mathrm{d}x = -\frac{1}{1 + \alpha^2}
%\]
%\end{example}

%=============================================================================
\section{Gauss 積分及其衍生}
%=============================================================================

Gauss 函數 $e^{-x^2}$ 是數學中最重要的函數之一。它出現在機率論、量子力學、統計力學、熱傳導與信號處理中。

\subsection{基本 Gauss 積分}

\begin{keyintegral}[Euler--Poisson 積分]
\[
\int_{-\infty}^{\infty} e^{-x^2}\,\mathrm{d}x = \sqrt{\pi}, \qquad \int_0^{\infty} e^{-x^2}\,\mathrm{d}x = \frac{\sqrt{\pi}}{2}
\]
\end{keyintegral}

\begin{prf}
令 $I = \int_0^{\infty} e^{-x^2}\,\mathrm{d}x$。我們將證明 $I = \frac{\sqrt{\pi}}{2}$。

\textbf{步驟 1:引入輔助函數。}

對於 $t \geqslant 0$,定義
\[
F(t) = \int_0^{\infty} \frac{e^{-t^2(1+x^2)}}{1+x^2}\,\mathrm{d}x
\]

\textbf{步驟 2:計算 $F(0)$ 和 $\lim_{t \to \infty} F(t)$。}

當 $t = 0$ 時,
\[
F(0) = \int_0^{\infty} \frac{1}{1+x^2}\,\mathrm{d}x = \bigl[\tan^{-1} x\bigr]_0^{\infty} = \frac{\pi}{2}
\]
當 $t \to \infty$ 時,被積函數 $\frac{e^{-t^2(1+x^2)}}{1+x^2} \leqslant e^{-t^2} \to 0$ 均勻收斂,故 $F(t) \to 0$。

\textbf{步驟 3:對 $F(t)$ 微分。}

我們應用定理~\ref{thm:leibniz-improper}。令 $f(x,t) = \frac{e^{-t^2(1+x^2)}}{1+x^2}$,其偏導數為
\[
\frac{\partial f}{\partial t} = \frac{-2t(1+x^2)e^{-t^2(1+x^2)}}{1+x^2} = -2t\, e^{-t^2(1+x^2)}
\]
對於 $t \in [0, T]$,我們有 $\left|\frac{\partial f}{\partial t}\right| \leqslant 2T e^{-t^2} \leqslant 2T$,但更有用的是,對於 $t \geqslant \varepsilon > 0$,
\[
\left|\frac{\partial f}{\partial t}\right| = 2t\, e^{-t^2(1+x^2)} \leqslant 2t\, e^{-\varepsilon^2(1+x^2)}
\]
在 $x$ 上可積。因此
\[
F'(t) = \int_0^{\infty} \frac{\partial}{\partial t}\left(\frac{e^{-t^2(1+x^2)}}{1+x^2}\right)\mathrm{d}x = -2t \int_0^{\infty} e^{-t^2(1+x^2)}\,\mathrm{d}x = -2t\, e^{-t^2} \int_0^{\infty} e^{-t^2 x^2}\,\mathrm{d}x
\]
代入 $u = tx$(故 $\mathrm{d}x = \mathrm{d}u/t$):
\[
F'(t) = -2t\, e^{-t^2} \cdot \frac{1}{t} \int_0^{\infty} e^{-u^2}\,\mathrm{d}u = -2\, e^{-t^2} \cdot I
\]

\textbf{步驟 4:將 $F'(t)$ 從 $0$ 積分到 $\infty$。}

\[
F(\infty) - F(0) = \int_0^{\infty} F'(t)\,\mathrm{d}t = -2I \int_0^{\infty} e^{-t^2}\,\mathrm{d}t = -2I \cdot I = -2I^2
\]
因此
\[
0 - \frac{\pi}{2} = -2I^2 \implies I^2 = \frac{\pi}{4} \implies I = \frac{\sqrt{\pi}}{2}
\]
(我們取正根,因為 $I > 0$。)
\end{prf}

\begin{application}[機率論:常態分佈]
均值為 $\mu$、變異數為 $\sigma^2$ 的常態分佈之機率密度函數為:
\[
f(x) = \frac{1}{\sigma\sqrt{2\pi}} \exp\left(-\frac{(x-\mu)^2}{2\sigma^2}\right)
\]
正規化條件 $\int_{-\infty}^{\infty} f(x)\,\mathrm{d}x = 1$ 來自:
\[
\int_{-\infty}^{\infty} e^{-(x-\mu)^2/(2\sigma^2)}\,\mathrm{d}x = \sigma\sqrt{2\pi}
\]
這可由 Gauss 積分透過代換 $u = (x-\mu)/(\sigma\sqrt{2})$ 得到。
\end{application}

\subsection{縮放的 Gauss 積分}

\begin{keyintegral}[縮放的 Gauss 積分]
對於 $a > 0$:
\[
\int_0^{\infty} e^{-ax^2}\,\mathrm{d}x = \frac{1}{2}\sqrt{\frac{\pi}{a}}
\]
\end{keyintegral}

\begin{prf}
代入 $u = \sqrt{a}\,x$,故 $\mathrm{d}u = \sqrt{a}\,\mathrm{d}x$:
\[
\int_0^{\infty} e^{-ax^2}\,\mathrm{d}x = \frac{1}{\sqrt{a}}\int_0^{\infty} e^{-u^2}\,\mathrm{d}u = \frac{1}{\sqrt{a}} \cdot \frac{\sqrt{\pi}}{2}.
\]
\end{prf}

\subsection{Gauss 分佈的動差:積分號下微分法}

\begin{keyintegral}[Gauss 動差]
對於 $a > 0$ 及 $n \in \N$:
\[
\int_0^{\infty} x^{2n} e^{-ax^2}\,\mathrm{d}x = \frac{(2n-1)!!}{2^{n+1}} \sqrt{\frac{\pi}{a^{2n+1}}} = \frac{(2n)!}{n! \, 4^n} \sqrt{\frac{\pi}{a^{2n+1}}}
\]
其中 $(2n-1)!! = 1 \cdot 3 \cdot 5 \cdots (2n-1)$ 是雙階乘。
\end{keyintegral}

\begin{prf}
令 $I(a) = \int_0^{\infty} e^{-ax^2}\,\mathrm{d}x = \frac{1}{2}\sqrt{\pi/a}$。

\textbf{微分合理性:}我們應用定理~\ref{thm:leibniz-improper},其中 $f(x,a) = e^{-ax^2}$。對於 $a \in [a_0, a_1]$($0 < a_0 < a_1$):
\begin{itemize}
\item $\frac{\partial f}{\partial a} = -x^2 e^{-ax^2}$ 對所有 $x \geqslant 0$ 及 $a > 0$ 存在。
\item $\left|\frac{\partial f}{\partial a}\right| = x^2 e^{-ax^2} \leqslant x^2 e^{-a_0 x^2}$,且 $\int_0^{\infty} x^2 e^{-a_0 x^2}\,\mathrm{d}x < \infty$。
\end{itemize}

因此積分號下微分是合理的:
\[
I'(a) = -\int_0^{\infty} x^2 e^{-ax^2}\,\mathrm{d}x = -\frac{1}{2}\sqrt{\pi} \cdot \left(-\frac{1}{2}\right) a^{-3/2} = \frac{\sqrt{\pi}}{4a^{3/2}}.
\]
故 $\int_0^{\infty} x^2 e^{-ax^2}\,\mathrm{d}x = \frac{\sqrt{\pi}}{4a^{3/2}}$。

對於更高階導數,同樣的論證適用,控制函數為 $x^{2n} e^{-a_0 x^2}$:
\[
I''(a) = \int_0^{\infty} x^4 e^{-ax^2}\,\mathrm{d}x = \frac{3\sqrt{\pi}}{8a^{5/2}}.
\]
一般而言,每次微分乘以 $(-1)$ 並引入 $x^2$:
\[
(-1)^n I^{(n)}(a) = \int_0^{\infty} x^{2n} e^{-ax^2}\,\mathrm{d}x
\]
由於 $\frac{\mathrm{d}^n}{\mathrm{d}a^n}\left(a^{-1/2}\right) = (-1)^n \frac{(2n-1)!!}{2^n} a^{-(2n+1)/2}$,公式得證。
\end{prf}

\begin{application}[統計力學:Maxwell--Boltzmann 分佈]
在溫度為 $T$ 的氣體中,分子速率 $v$ 的機率分佈為:
\[
f(v) = 4\pi n \left(\frac{m}{2\pi k_B T}\right)^{3/2} v^2 \exp\left(-\frac{mv^2}{2k_B T}\right)
\]
均方速率為:
\[
\langle v^2 \rangle = \frac{\int_0^{\infty} v^4 e^{-mv^2/(2k_BT)}\,\mathrm{d}v}{\int_0^{\infty} v^2 e^{-mv^2/(2k_BT)}\,\mathrm{d}v} = \frac{3k_B T}{m}
\]
這給出能量均分定理:$\langle \frac{1}{2}mv^2 \rangle = \frac{3}{2}k_B T$。
\end{application}

\begin{application}[量子力學:諧振子]
量子諧振子的基態波函數為:
\[
\psi_0(x) = \left(\frac{m\omega}{\pi\hbar}\right)^{1/4} \exp\left(-\frac{m\omega x^2}{2\hbar}\right)
\]
位置不確定性為:
\[
\langle x^2 \rangle = \int_{-\infty}^{\infty} x^2 |\psi_0(x)|^2\,\mathrm{d}x = \sqrt{\frac{m\omega}{\pi\hbar}} \int_{-\infty}^{\infty} x^2 e^{-m\omega x^2/\hbar}\,\mathrm{d}x = \frac{\hbar}{2m\omega}
\]
結合 $\langle p^2 \rangle = \frac{m\omega\hbar}{2}$,這給出不確定性乘積 $\Delta x \cdot \Delta p = \frac{\hbar}{2}$。
\end{application}

\subsection{Gauss Fourier 變換}

\begin{keyintegral}[Gauss Fourier 變換]
對於 $a > 0$:
\[
\int_0^{\infty} e^{-ax^2}\cos bx\,\mathrm{d}x = \frac{1}{2}\sqrt{\frac{\pi}{a}}\, e^{-b^2/(4a)}
\]
\end{keyintegral}

\begin{prf}
定義 $I(b) = \int_0^{\infty} e^{-ax^2}\cos(bx)\,\mathrm{d}x$,其中 $\ds I(0) = \frac{1}{2}\sqrt{\pi/a}$。

\textbf{微分合理性:}我們應用定理~\ref{thm:leibniz-improper},其中 $f(x,b) = e^{-ax^2}\cos(bx)$。偏導數為 $\frac{\partial f}{\partial b} = -x e^{-ax^2}\sin(bx)$。對所有 $b \in \R$:
\[
\left|\frac{\partial f}{\partial b}\right| = |x| e^{-ax^2} |\sin(bx)| \leqslant x e^{-ax^2}
\]
且 $\int_0^{\infty} x e^{-ax^2}\,\mathrm{d}x = \frac{1}{2a} < \infty$。因此微分是合理的。

微分:
\[
I'(b) = -\int_0^{\infty} x\, e^{-ax^2}\sin(bx)\,\mathrm{d}x.
\]
使用分部積分,令 $u = \sin(bx)$,$\mathrm{d}v = -xe^{-ax^2}\,\mathrm{d}x$,故 $v = \frac{1}{2a}e^{-ax^2}$:
\[
I'(b) = \left[\frac{\sin(bx)}{2a}e^{-ax^2}\right]_0^{\infty} - \frac{b}{2a}\int_0^{\infty} e^{-ax^2}\cos(bx)\,\mathrm{d}x = 0 - \frac{b}{2a}I(b).
\]
此常微分方程 $I'(b) = -\frac{b}{2a}I(b)$ 的解為 $I(b) = I(0)e^{-b^2/(4a)}$。
\end{prf}

\begin{application}[信號處理:Gauss 脈衝傳播]
時域中的 Gauss 脈衝 $f(t) = e^{-at^2}$ 的 Fourier 變換為:
\[
\hat{f}(\omega) = \int_{-\infty}^{\infty} e^{-at^2} e^{-i\omega t}\,\mathrm{d}t = \sqrt{\frac{\pi}{a}} e^{-\omega^2/(4a)}
\]
變換也是 Gauss 函數。時間-頻寬乘積滿足:
\[
\Delta t \cdot \Delta \omega = \frac{1}{2\sqrt{a}} \cdot \sqrt{a} = \frac{1}{2}
\]
這是最小不確定性,使 Gauss 脈衝成為通訊系統的最佳選擇。
\end{application}

\subsection{Glasser 積分}

\begin{keyintegral}[Glasser 主定理的應用]
對於 $a > 0$:
\[
\int_0^{\infty} e^{-x^2 - a^2/x^2}\,\mathrm{d}x = \frac{\sqrt{\pi}}{2}e^{-2a}
\]
\end{keyintegral}

\begin{prf}
令 $I(a) = \int_0^{\infty} e^{-x^2 - a^2/x^2}\,\mathrm{d}x$。

\textbf{微分合理性:}我們應用定理~\ref{thm:leibniz-improper},其中 $f(x,a) = e^{-x^2 - a^2/x^2}$。偏導數為:
\[
\frac{\partial f}{\partial a} = -\frac{2a}{x^2} e^{-x^2 - a^2/x^2}
\]
對於 $a \in [0, A]$ 及 $x > 0$,我們需要控制函數。注意 $e^{-x^2 - a^2/x^2} \leqslant e^{-x^2}$ 且 $\frac{2a}{x^2} \leqslant \frac{2A}{x^2}$。對於 $x \geqslant 1$:$\left|\frac{\partial f}{\partial a}\right| \leqslant 2A e^{-x^2}$。對於 $x \in (0,1)$:$\left|\frac{\partial f}{\partial a}\right| \leqslant \frac{2A}{x^2} e^{-a^2/x^2}$,積分收斂。因此微分是合理的。

微分:
\[
I'(a) = -2a \int_0^{\infty} \frac{e^{-x^2 - a^2/x^2}}{x^2}\,\mathrm{d}x.
\]
代入 $u = a/x$(故 $x = a/u$,$\mathrm{d}x = -a\,\mathrm{d}u/u^2$):
\[
\int_0^{\infty} \frac{e^{-x^2 - a^2/x^2}}{x^2}\,\mathrm{d}x = \frac{1}{a}\int_0^{\infty} e^{-a^2/u^2 - u^2}\,\mathrm{d}u = \frac{I(a)}{a}.
\]
因此 $I'(a) = -2I(a)$,得 $I(a) = I(0)e^{-2a} = \frac{\sqrt{\pi}}{2}e^{-2a}$。
\end{prf}

\begin{application}[量子場論:路徑積分]
在量子力學的路徑積分表述中,自由粒子的傳播子涉及:
\[
K(x_f, t_f; x_i, t_i) = \sqrt{\frac{m}{2\pi i\hbar(t_f - t_i)}} \exp\left(\frac{im(x_f - x_i)^2}{2\hbar(t_f - t_i)}\right)
\]
當計算具有特定函數形式的動能和位能貢獻的系統傳播子時,會出現 Glasser 型積分。
\end{application}

%=============================================================================
\section{Dirichlet 積分與信號處理}
%=============================================================================

\subsection{Sinc 函數}

函數 $\mathrm{sinc}(x) = \frac{\sin x}{x}$(其中 $\mathrm{sinc}(0) = 1$)在信號處理中是基本的。

\begin{keyintegral}[Dirichlet 積分]
\[
\int_0^{\infty} \frac{\sin x}{x}\,\mathrm{d}x = \frac{\pi}{2}
\]
\end{keyintegral}

\begin{prf}
引入收斂因子 $e^{-ax}$:
\[
I(a) = \int_0^{\infty} \frac{\sin x}{x} e^{-ax}\,\mathrm{d}x, \quad a > 0.
\]

\textbf{微分合理性:}我們應用定理~\ref{thm:leibniz-improper},其中 $f(x,a) = \frac{\sin x}{x} e^{-ax}$。偏導數為 $\frac{\partial f}{\partial a} = -\sin x \cdot e^{-ax}$。對於 $a \geqslant a_0 > 0$:
\[
\left|\frac{\partial f}{\partial a}\right| = |\sin x| e^{-ax} \leqslant e^{-a_0 x}
\]
且 $\int_0^{\infty} e^{-a_0 x}\,\mathrm{d}x = \frac{1}{a_0} < \infty$。因此微分是合理的。

微分:
\[
I'(a) = -\int_0^{\infty} \sin x \cdot e^{-ax}\,\mathrm{d}x.
\]

\textbf{計算 $\int_0^{\infty} e^{-ax}\sin x\,\mathrm{d}x$:}

分部積分兩次。令 $J = \int_0^{\infty} e^{-ax}\sin x\,\mathrm{d}x$。

第一次:$u = \sin x$,$\mathrm{d}v = e^{-ax}\,\mathrm{d}x$,得 $v = -\frac{1}{a}e^{-ax}$:
\[
J = \left[-\frac{\sin x}{a}e^{-ax}\right]_0^{\infty} + \frac{1}{a}\int_0^{\infty} e^{-ax}\cos x\,\mathrm{d}x = \frac{1}{a}\int_0^{\infty} e^{-ax}\cos x\,\mathrm{d}x.
\]

第二次:$u = \cos x$,$\mathrm{d}v = e^{-ax}\,\mathrm{d}x$:
\[
J = \frac{1}{a}\left(\left[-\frac{\cos x}{a}e^{-ax}\right]_0^{\infty} - \frac{1}{a}\int_0^{\infty} e^{-ax}\sin x\,\mathrm{d}x\right) = \frac{1}{a}\left(\frac{1}{a} - \frac{J}{a}\right).
\]

因此 $J = \frac{1}{a^2} - \frac{J}{a^2}$,得 $J\left(1 + \frac{1}{a^2}\right) = \frac{1}{a^2}$,故:
\[
J = \frac{1}{a^2 + 1}.
\]

因此 $I'(a) = -\frac{1}{1+a^2}$。

積分:$I(a) = -\tan^{-1} a + C$。

當 $a \to \infty$ 時,$I(a) \to 0$(被積函數被 $e^{-ax}/x$ 控制而趨於零),故 $C = \frac{\pi}{2}$。因此 $I(a) = \frac{\pi}{2} - \tan^{-1} a$。

取 $a \to 0^+$:
\[
\int_0^{\infty} \frac{\sin x}{x}\,\mathrm{d}x = \frac{\pi}{2}.
\]
\end{prf}

\begin{application}[信號處理:理想低通濾波器]
截止頻率為 $\omega_c$ 的理想低通濾波器具有頻率響應:
\[
H(\omega) = \begin{cases} 1 & |\omega| \leqslant \omega_c \\ 0 & |\omega| > \omega_c \end{cases}
\]
其脈衝響應(逆 Fourier 變換)為:
\[
h(t) = \frac{1}{2\pi}\int_{-\omega_c}^{\omega_c} e^{i\omega t}\,\mathrm{d}\omega = \frac{\omega_c}{\pi} \cdot \frac{\sin(\omega_c t)}{\omega_c t} = \frac{\omega_c}{\pi}\mathrm{sinc}(\omega_c t)
\]
$h(t)$ 下的總面積等於 $H(0) = 1$:
\[
\int_{-\infty}^{\infty} h(t)\,\mathrm{d}t = \frac{2\omega_c}{\pi} \int_0^{\infty} \frac{\sin(\omega_c t)}{\omega_c t}\,\mathrm{d}t = \frac{2}{\pi} \cdot \frac{\pi}{2} = 1
\]
\end{application}

\subsection{廣義 Sinc 積分}

\begin{keyintegral}[衰減的 Sinc]
對於 $a > 0$:
\[
\int_0^{\infty} e^{-ax}\frac{\sin(bx)}{x}\,\mathrm{d}x = \tan^{-1}\frac{b}{a}
\]
\end{keyintegral}

\begin{prf}
令 $I(b) = \int_0^{\infty} e^{-ax}\frac{\sin(bx)}{x}\,\mathrm{d}x$,其中 $I(0) = 0$。

\textbf{微分合理性:}我們應用定理~\ref{thm:leibniz-improper}。偏導數為 $\frac{\partial}{\partial b}\left(e^{-ax}\frac{\sin(bx)}{x}\right) = e^{-ax}\cos(bx)$。對所有 $b$:
\[
\left|e^{-ax}\cos(bx)\right| \leqslant e^{-ax}
\]
且 $\int_0^{\infty} e^{-ax}\,\mathrm{d}x = \frac{1}{a} < \infty$。因此微分是合理的。

則 $I'(b) = \int_0^{\infty} e^{-ax}\cos(bx)\,\mathrm{d}x$。

使用與上面相同的分部積分技巧(或認出這是 $\mathrm{Re}\int_0^{\infty} e^{-(a-ib)x}\,\mathrm{d}x$):
\[
I'(b) = \frac{a}{a^2+b^2}.
\]

積分:$I(b) = \tan^{-1}(b/a)$。
\end{prf}

\begin{keyintegral}[Sinc 平方]
對於 $a > 0$:
\[
\int_0^{\infty} \frac{\sin^2(ax)}{x^2}\,\mathrm{d}x = \frac{\pi a}{2}
\]
\end{keyintegral}

\begin{prf}
令 $I(a) = \int_0^{\infty} \frac{\sin^2(ax)}{x^2}\,\mathrm{d}x$,其中 $I(0) = 0$。

\textbf{微分合理性:}我們在 $[0, R]$ 上應用定理~\ref{thm:leibniz-finite},然後取 $R \to \infty$。或者,注意 $\frac{\partial}{\partial a}\frac{\sin^2(ax)}{x^2} = \frac{2\sin(ax)\cos(ax) \cdot x}{x^2} = \frac{\sin(2ax)}{x}$。對於任意 $a \in [0, A]$,當 $x \geqslant 1$ 時 $\left|\frac{\sin(2ax)}{x}\right| \leqslant \frac{1}{x}$,在 $0$ 附近有界。積分的收斂由 Dirichlet 積分理論保證。

使用 $\frac{\partial}{\partial a}\sin^2(ax) = 2\sin(ax)\cos(ax) \cdot x = x\sin(2ax)$ 微分:
\[
I'(a) = \int_0^{\infty} \frac{\sin(2ax)}{x}\,\mathrm{d}x.
\]

代入 $u = 2ax$:
\[
I'(a) = \int_0^{\infty} \frac{\sin u}{u}\,\mathrm{d}u = \frac{\pi}{2}.
\]

因此 $I(a) = \frac{\pi a}{2}$。
\end{prf}

\begin{application}[光學:單狹縫 Fraunhofer 繞射]
對於寬度為 $b$ 的狹縫,被波長為 $\lambda$ 的單色光照射,在角度 $\theta$ 處的強度圖案為:
\[
I(\theta) = I_0 \left(\frac{\sin\beta}{\beta}\right)^2, \quad \text{其中 } \beta = \frac{\pi b \sin\theta}{\lambda}
\]
通過狹縫傳輸的總功率正比於:
\[
\int_{-\pi/2}^{\pi/2} I(\theta)\cos\theta\,\mathrm{d}\theta \approx I_0 \int_{-\infty}^{\infty} \mathrm{sinc}^2\left(\frac{\pi b \theta}{\lambda}\right)\mathrm{d}\theta = I_0 \cdot \frac{\lambda}{b}
\]
(使用小角度近似 $\sin\theta \approx \theta$)。
\end{application}

\subsection{餘弦差積分}

\begin{keyintegral}
對於 $a, b > 0$:
\[
\int_0^{\infty} \frac{\cos(ax) - \cos(bx)}{x^2}\,\mathrm{d}x = \frac{\pi(b-a)}{2}
\]
\end{keyintegral}

\begin{prf}
定義 $J(a) = \int_0^{\infty} \frac{1 - \cos(ax)}{x^2}\,\mathrm{d}x$,其中 $J(0) = 0$。

\textbf{微分合理性:}被積函數 $\frac{1-\cos(ax)}{x^2}$ 在 $x = 0$ 附近表現為 $\frac{a^2}{2}$,對於大 $x$ 表現為 $\frac{1}{x^2}$。偏導數 $\frac{\partial}{\partial a}\frac{1-\cos(ax)}{x^2} = \frac{\sin(ax)}{x}$。由 Dirichlet 積分收斂性,微分是合理的。

則 $J'(a) = \int_0^{\infty} \frac{\sin(ax)}{x}\,\mathrm{d}x = \frac{\pi}{2}$。

故 $J(a) = \frac{\pi a}{2}$,結果由 $J(b) - J(a)$ 得到。
\end{prf}

%=============================================================================
\section{Frullani 積分}
%=============================================================================

\begin{keyintegral}[Frullani 定理]
若 $f$ 在 $(0, \infty)$ 上連續,且 $f(0^+)$ 和 $f(\infty) = \lim_{x\to\infty} f(x)$ 都存在且有限,則對於 $a, b > 0$:
\[
\int_0^{\infty} \frac{f(ax) - f(bx)}{x}\,\mathrm{d}x = \bigl(f(0^+) - f(\infty)\bigr) \ln\frac{b}{a}
\]
\end{keyintegral}

\begin{prf}
定義 $I(a) = \int_0^{\infty} \frac{f(ax) - f(bx)}{x}\,\mathrm{d}x$。注意 $I(b) = 0$。

\textbf{微分合理性:}我們使用極限論證。當 $f$ 可微時,導數 $\frac{\partial}{\partial a}\frac{f(ax) - f(bx)}{x} = f'(ax)$。對於足夠正則的 $f$,控制收斂定理適用。

微分:
\[
I'(a) = \int_0^{\infty} f'(ax)\,\mathrm{d}x.
\]

代入 $u = ax$:
\[
I'(a) = \frac{1}{a}\int_0^{\infty} f'(u)\,\mathrm{d}u = \frac{1}{a}\bigl[f(\infty) - f(0^+)\bigr].
\]

從 $b$ 積分到 $a$:
\[
I(a) - I(b) = \bigl(f(\infty) - f(0^+)\bigr)\ln\frac{a}{b}.
\]
由於 $I(b) = 0$:
\[
I(a) = \bigl(f(\infty) - f(0^+)\bigr)\ln\frac{a}{b} = \bigl(f(0^+) - f(\infty)\bigr)\ln\frac{b}{a}.
\]
\end{prf}

\begin{keyintegral}[指數 Frullani]
\[
\int_0^{\infty} \frac{e^{-ax} - e^{-bx}}{x}\,\mathrm{d}x = \ln\frac{b}{a}
\]
\end{keyintegral}

這裡 $f(x) = e^{-x}$,其中 $f(0^+) = 1$,$f(\infty) = 0$。

\begin{keyintegral}[反正切 Frullani]
\[
\int_0^{\infty} \frac{\tan^{-1}(ax) - \tan^{-1}(bx)}{x}\,\mathrm{d}x = \frac{\pi}{2}\ln\frac{a}{b}
\]
\end{keyintegral}

這裡 $f(x) = \tan^{-1}(x)$,其中 $f(0^+) = 0$,$f(\infty) = \pi/2$。

\begin{application}[量子電動力學:正則化]
在量子場論中,經常出現發散積分。Frullani 技巧提供了一種自然的正則化方法:不是計算 $\int_0^{\infty} \frac{f(ax)}{x}\,\mathrm{d}x$(可能發散),而是計算:
\[
\int_0^{\infty} \frac{f(ax) - f(bx)}{x}\,\mathrm{d}x = \bigl(f(0^+) - f(\infty)\bigr)\ln\frac{b}{a}
\]
參數 $b$ 充當調節器,在重整化後取 $b \to 0$ 或 $b \to \infty$ 的極限得到物理結果。
\end{application}

%=============================================================================
\section{對數積分}
%=============================================================================

\subsection{基本對數積分}

\begin{keyintegral}
對於 $a > -1$:
\[
\int_0^1 \frac{x^a - 1}{\ln x}\,\mathrm{d}x = \ln(a + 1)
\]
\end{keyintegral}

\begin{prf}
令 $I(a) = \int_0^1 \frac{x^a - 1}{\ln x}\,\mathrm{d}x$,其中 $I(0) = 0$。

\textbf{微分合理性:}我們應用定理~\ref{thm:leibniz-finite}。被積函數 $f(x,a) = \frac{x^a - 1}{\ln x}$ 在 $(0,1]$ 上連續,在 $x = 1$ 有可去奇點(分子和分母都為零)。偏導數為 $\frac{\partial f}{\partial a} = \frac{x^a \ln x}{\ln x} = x^a$。對於 $a \in [a_0, a_1]$($a_0 > -1$),$|x^a| \leqslant \max(x^{a_0}, x^{a_1})$,在 $[0,1]$ 上可積。因此微分是合理的。

微分:$I'(a) = \int_0^1 \frac{x^a \ln x}{\ln x}\,\mathrm{d}x = \int_0^1 x^a\,\mathrm{d}x = \frac{1}{a+1}$。

積分:$I(a) = \ln(a+1)$。
\end{prf}

\begin{keyintegral}[廣義形式]
對於 $a, b > -1$:
\[
\int_0^1 \frac{x^a - x^b}{\ln x}\,\mathrm{d}x = \ln\frac{a+1}{b+1}
\]
\end{keyintegral}

\begin{application}[資訊理論:熵計算]
連續隨機變數 $X$ 具有密度 $p(x)$ 的微分熵為:
\[
H(X) = -\int p(x) \ln p(x)\,\mathrm{d}x
\]
對於 $[0,1]$ 上的冪律分佈 $p(x) \propto x^a$,熵涉及:
\[
\int_0^1 x^a \ln x\,\mathrm{d}x = -\frac{1}{(a+1)^2}
\]
這可以透過對 $\int_0^1 x^a\,\mathrm{d}x = \frac{1}{a+1}$ 關於 $a$ 微分得到。
\end{application}

\subsection{對數-有理積分}

\begin{keyintegral}
對於 $a > 0$:
\[
\int_0^{\infty} \frac{\ln x}{x^2 + a^2}\,\mathrm{d}x = \frac{\pi \ln a}{2a}
\]
\end{keyintegral}

\begin{prf}
令 $I(a) = \int_0^{\infty} \frac{\ln x}{x^2 + a^2}\,\mathrm{d}x$。代入 $x = au$:
\[
I(a) = \int_0^{\infty} \frac{\ln(au)}{a^2 u^2 + a^2}\cdot a\, \mathrm{d}u = \frac{1}{a}\int_0^{\infty} \frac{\ln a + \ln u}{u^2 + 1}\,\mathrm{d}u.
\]

現在 $\int_0^{\infty} \frac{\mathrm{d}u}{u^2+1} = \frac{\pi}{2}$。

對於 $\int_0^{\infty} \frac{\ln u}{u^2+1}\,\mathrm{d}u$,代入 $u = 1/t$:
\[
\int_0^{\infty} \frac{\ln u}{u^2+1}\,\mathrm{d}u = \int_{\infty}^{0} \frac{-\ln t}{1/t^2+1}\cdot\frac{-\mathrm{d}t}{t^2} = \int_0^{\infty} \frac{-\ln t}{t^2+1}\,\mathrm{d}t.
\]

因此 $\int_0^{\infty} \frac{\ln u}{u^2+1}\,\mathrm{d}u = -\int_0^{\infty} \frac{\ln u}{u^2+1}\,\mathrm{d}u$,故此積分等於 $0$。

因此 $I(a) = \frac{\ln a}{a} \cdot \frac{\pi}{2} = \frac{\pi \ln a}{2a}$。
\end{prf}

\begin{application}[靜電學:二維 Green 函數]
在二維中,線電荷產生的靜電位滿足:
\[
\nabla^2 \phi = -\frac{\lambda}{\varepsilon_0}\delta^{(2)}(\mathbf{r})
\]
Green 函數為 $G(\mathbf{r}, \mathbf{r}') = -\frac{1}{2\pi}\ln|\mathbf{r} - \mathbf{r}'|$。計算電荷分佈的電位涉及如下積分:
\[
\phi(\mathbf{r}) = \frac{\lambda}{2\pi\varepsilon_0}\int \ln|\mathbf{r} - \mathbf{r}'|\,\mathrm{d}\ell'
\]
對數-有理積分公式有助於計算特定幾何的電位。
\end{application}

\subsection{三角-對數積分}

\begin{keyintegral}
對於 $a, b > 0$:
\[
\int_0^{\pi/2} \ln(a^2\cos^2\theta + b^2\sin^2\theta)\,\mathrm{d}\theta = \pi\ln\frac{a+b}{2}
\]
\end{keyintegral}

\begin{prf}
令 $I(a,b) = \int_0^{\pi/2} \ln(a^2\cos^2\theta + b^2\sin^2\theta)\,\mathrm{d}\theta$。

\textbf{步驟 1:計算 $\frac{\partial I}{\partial a}$。}

由定理~\ref{thm:leibniz-finite},因為被積函數及其偏導數在 $[0, \pi/2]$ 上對於 $a, b > 0$ 連續:
\[
\frac{\partial I}{\partial a} = \int_0^{\pi/2} \frac{2a\cos^2\theta}{a^2\cos^2\theta + b^2\sin^2\theta}\,\mathrm{d}\theta
\]

代入 $t = \tan\theta$,故 $\cos^2\theta = \frac{1}{1+t^2}$,$\sin^2\theta = \frac{t^2}{1+t^2}$,$\mathrm{d}\theta = \frac{\mathrm{d}t}{1+t^2}$:
\[
\frac{\partial I}{\partial a} = \int_0^{\infty} \frac{2a/(1+t^2)}{(a^2 + b^2 t^2)/(1+t^2)} \cdot \frac{\mathrm{d}t}{1+t^2} = \int_0^{\infty} \frac{2a}{(a^2 + b^2 t^2)(1+t^2)}\,\mathrm{d}t
\]

\textbf{步驟 2:部分分式。}

對於 $a \neq b$:
\[
\frac{1}{(a^2 + b^2 t^2)(1+t^2)} = \frac{1}{a^2-b^2}\left(\frac{1}{1+t^2} - \frac{b^2}{a^2+b^2t^2}\right)
\]

因此:
\[
\frac{\partial I}{\partial a} = \frac{2a}{a^2-b^2}\left(\int_0^{\infty}\frac{\mathrm{d}t}{1+t^2} - b^2\int_0^{\infty}\frac{\mathrm{d}t}{a^2+b^2t^2}\right)
\]
\[
= \frac{2a}{a^2-b^2}\left(\frac{\pi}{2} - b^2 \cdot \frac{1}{ab} \cdot \frac{\pi}{2}\right) = \frac{2a}{a^2-b^2} \cdot \frac{\pi}{2} \cdot \frac{a-b}{a} = \frac{\pi}{a+b}
\]

由對稱性,$\frac{\partial I}{\partial b} = \frac{\pi}{a+b}$。

\textbf{步驟 3:積分偏導數。}

兩個偏導數都等於 $\frac{\pi}{a+b}$,這與以下一致:
\[
I(a,b) = \pi\ln(a+b) + C
\]

\textbf{步驟 4:確定 $C$。}

令 $a = b$:
\[
I(a,a) = \int_0^{\pi/2} \ln(a^2)\,\mathrm{d}\theta = \pi\ln a
\]

但從我們的公式:$\pi\ln(2a) + C = \pi\ln a$,故 $C = -\pi\ln 2$。

因此:
\[
I(a,b) = \pi\ln(a+b) - \pi\ln 2 = \pi\ln\frac{a+b}{2}
\]
\end{prf}

\begin{keyintegral}[對數-正弦和對數-餘弦]
\[
\int_0^{\pi/2} \ln(\sin\theta)\,\mathrm{d}\theta = \int_0^{\pi/2} \ln(\cos\theta)\,\mathrm{d}\theta = -\frac{\pi}{2}\ln 2
\]
\end{keyintegral}

\begin{prf}
令 $I = \int_0^{\pi/2} \ln(\sin\theta)\,\mathrm{d}\theta$。代入 $\phi = \frac{\pi}{2} - \theta$:
\[
I = \int_0^{\pi/2} \ln(\cos\phi)\,\mathrm{d}\phi.
\]
故兩個積分相等。

現在計算 $2I = \int_0^{\pi/2} \ln(\sin\theta\cos\theta)\,\mathrm{d}\theta = \int_0^{\pi/2} \ln\frac{\sin 2\theta}{2}\,\mathrm{d}\theta$。

\[
2I = \int_0^{\pi/2} \ln(\sin 2\theta)\,\mathrm{d}\theta - \frac{\pi}{2}\ln 2.
\]

代入 $u = 2\theta$:
\[
\int_0^{\pi/2} \ln(\sin 2\theta)\,\mathrm{d}\theta = \frac{1}{2}\int_0^{\pi} \ln(\sin u)\,\mathrm{d}u = \frac{1}{2} \cdot 2\int_0^{\pi/2}\ln(\sin u)\,\mathrm{d}u = I.
\]

(我們使用了 $\int_0^{\pi}\ln\sin u\,\mathrm{d}u = 2\int_0^{\pi/2}\ln\sin u\,\mathrm{d}u$,因為對 $\pi/2$ 對稱。)

因此 $2I = I - \frac{\pi}{2}\ln 2$,得 $I = -\frac{\pi}{2}\ln 2$。
\end{prf}

\begin{application}[統計力學:態密度]
在二維 Ising 模型中,配分函數涉及如下形式的積分:
\[
\ln Z = \frac{1}{2}\ln 2 + \frac{1}{2\pi^2}\int_0^{\pi}\int_0^{\pi} \ln\bigl[\cosh^2(2K) - \sinh(2K)(\cos\phi_1 + \cos\phi_2)\bigr]\,\mathrm{d}\phi_1\,\mathrm{d}\phi_2
\]
對數-正弦積分出現在這類表達式的計算中,特別是在 Onsager 的精確解中。
\end{application}

%=============================================================================
\section{Poisson 積分}
%=============================================================================

\begin{keyintegral}[Poisson 積分]
對於 $|r| < 1$:
\[
\int_0^\pi \ln(1 - 2r\cos\theta + r^2)\,\mathrm{d}\theta = 0
\]
對於 $|r| > 1$:
\[
\int_0^\pi \ln(1 - 2r\cos\theta + r^2)\,\mathrm{d}\theta = 2\pi\ln|r|
\]
\end{keyintegral}

\begin{prf}
定義 $I(r) = \int_0^{\pi} \ln(1 - 2r\cos\theta + r^2)\,\mathrm{d}\theta$,對於 $|r| < 1$。

注意 $I(0) = \int_0^{\pi} \ln(1)\,\mathrm{d}\theta = 0$。

\textbf{步驟 1:對 $r$ 微分。}

由定理~\ref{thm:leibniz-finite},因為被積函數及其偏導數在 $[0,\pi]$ 上對於 $|r| < 1$ 連續:
\[
I'(r) = \int_0^\pi \frac{-2\cos\theta + 2r}{1 - 2r\cos\theta + r^2}\,\mathrm{d}\theta = 2r\int_0^\pi \frac{\mathrm{d}\theta}{1 - 2r\cos\theta + r^2} - 2\int_0^\pi \frac{\cos\theta\,\mathrm{d}\theta}{1 - 2r\cos\theta + r^2}
\]

\textbf{步驟 2:計算 $\int_0^\pi \frac{\mathrm{d}\theta}{1 - 2r\cos\theta + r^2}$。}

使用 Weierstrass 代換 $t = \tan(\theta/2)$,故:
\begin{align*}
\cos\theta &= \frac{1-t^2}{1+t^2}, \quad \mathrm{d}\theta = \frac{2\,\mathrm{d}t}{1+t^2}
\end{align*}

分母變換為:
\[
1 - 2r\cos\theta + r^2 = 1 - 2r\frac{1-t^2}{1+t^2} + r^2 = \frac{(1-r)^2 + (1+r)^2 t^2}{1+t^2}
\]

因此:
\[
\int_0^\pi \frac{\mathrm{d}\theta}{1 - 2r\cos\theta + r^2} = \int_0^\infty \frac{1+t^2}{(1-r)^2 + (1+r)^2 t^2} \cdot \frac{2\,\mathrm{d}t}{1+t^2} = 2\int_0^\infty \frac{\mathrm{d}t}{(1-r)^2 + (1+r)^2 t^2}
\]

代入 $u = (1+r)t$:
\[
= \frac{2}{1+r}\int_0^\infty \frac{\mathrm{d}u}{(1-r)^2 + u^2} = \frac{2}{1+r} \cdot \frac{1}{1-r} \cdot \frac{\pi}{2} = \frac{\pi}{1-r^2}
\]

\textbf{步驟 3:計算 $\int_0^\pi \frac{\cos\theta\,\mathrm{d}\theta}{1 - 2r\cos\theta + r^2}$。}

使用相同的代換:
\[
\int_0^\pi \frac{\cos\theta\,\mathrm{d}\theta}{1 - 2r\cos\theta + r^2} = 2\int_0^\infty \frac{1-t^2}{[(1-r)^2 + (1+r)^2 t^2](1+t^2)}\,\mathrm{d}t
\]

令 $\alpha = (1-r)^2$ 和 $\beta = (1+r)^2$。使用部分分式:
\[
\frac{1-t^2}{(\alpha+\beta t^2)(1+t^2)} = \frac{\alpha+\beta}{(\beta-\alpha)(\alpha+\beta t^2)} - \frac{2}{(\beta-\alpha)(1+t^2)}
\]

現在 $\beta - \alpha = (1+r)^2 - (1-r)^2 = 4r$ 且 $\alpha + \beta = 2(1+r^2)$。

因此:
\[
\int_0^\infty \frac{1-t^2}{(\alpha+\beta t^2)(1+t^2)}\,\mathrm{d}t = \frac{1+r^2}{2r}\cdot\frac{\pi}{2(1-r^2)} - \frac{1}{2r}\cdot\frac{\pi}{2} = \frac{\pi(1+r^2)}{4r(1-r^2)} - \frac{\pi}{4r}
\]
\[
= \frac{\pi}{4r}\left(\frac{1+r^2}{1-r^2} - 1\right) = \frac{\pi}{4r}\cdot\frac{2r^2}{1-r^2} = \frac{\pi r}{2(1-r^2)}
\]

故:
\[
\int_0^\pi \frac{\cos\theta\,\mathrm{d}\theta}{1 - 2r\cos\theta + r^2} = 2 \cdot \frac{\pi r}{2(1-r^2)} = \frac{\pi r}{1-r^2}
\]

\textbf{步驟 4:合併。}
\[
I'(r) = 2r \cdot \frac{\pi}{1-r^2} - 2 \cdot \frac{\pi r}{1-r^2} = \frac{2\pi r - 2\pi r}{1-r^2} = 0
\]

\textbf{步驟 5:結論。}

由於對所有 $|r| < 1$ 有 $I'(r) = 0$ 且 $I(0) = 0$,我們有對於 $|r| < 1$,$I(r) = 0$。

\textbf{情況 $|r| > 1$:}

令 $r = 1/s$,其中 $|s| < 1$。則:
\[
1 - 2r\cos\theta + r^2 = r^2\left(s^2 - 2s\cos\theta + 1\right) = r^2(1 - 2s\cos\theta + s^2)
\]

故:
\[
I(r) = \int_0^{\pi} \ln\bigl[r^2(1 - 2s\cos\theta + s^2)\bigr]\,\mathrm{d}\theta = 2\pi\ln|r| + \underbrace{\int_0^{\pi}\ln(1-2s\cos\theta+s^2)\,\mathrm{d}\theta}_{=0} = 2\pi\ln|r|
\]
\end{prf}

\begin{application}[位勢理論:調和函數]
對於單位圓盤內的調和函數 $u(r, \theta)$,給定圓上的邊界值 $f(\phi)$,Poisson 積分公式為:
\[
u(r, \theta) = \frac{1}{2\pi}\int_0^{2\pi} \frac{1 - r^2}{1 - 2r\cos(\theta - \phi) + r^2} f(\phi)\,\mathrm{d}\phi
\]
核 $P_r(\theta) = \frac{1-r^2}{1-2r\cos\theta+r^2}$ 稱為 Poisson 核。我們證明的對數積分表明:
\[
\int_0^{2\pi} \ln|1 - re^{i\theta}|^2\,\mathrm{d}\theta = 0 \quad \text{對於 } |r| < 1
\]
這與複分析中的 Jensen 公式有關。
\end{application}

%=============================================================================
\section{反正切積分}
%=============================================================================

\begin{keyintegral}
對於 $a > 0$:
\[
\int_0^{\infty} \frac{\tan^{-1}(ax)}{x(1 + x^2)}\,\mathrm{d}x = \frac{\pi}{2}\ln(1 + a)
\]
\end{keyintegral}

\begin{prf}
令 $I(a) = \int_0^{\infty} \frac{\tan^{-1}(ax)}{x(1+x^2)}\,\mathrm{d}x$,其中 $I(0) = 0$。

\textbf{微分合理性:}我們應用定理~\ref{thm:leibniz-improper}。偏導數為:
\[
\frac{\partial}{\partial a}\frac{\tan^{-1}(ax)}{x(1+x^2)} = \frac{1}{(1+a^2x^2)(1+x^2)}
\]
對於 $a \in [0, A]$,我們有 $\frac{1}{(1+a^2x^2)(1+x^2)} \leqslant \frac{1}{1+x^2}$,且 $\int_0^{\infty} \frac{\mathrm{d}x}{1+x^2} = \frac{\pi}{2} < \infty$。因此微分是合理的。

微分:
\[
I'(a) = \int_0^{\infty} \frac{1}{(1 + a^2 x^2)(1 + x^2)}\,\mathrm{d}x.
\]
部分分式(對於 $a \neq 1$):
\[
\frac{1}{(1+a^2x^2)(1+x^2)} = \frac{1}{1-a^2}\left(\frac{1}{1+x^2} - \frac{a^2}{1+a^2x^2}\right).
\]
故
\[
I'(a) = \frac{1}{1-a^2}\left(\int_0^{\infty}\frac{\mathrm{d}x}{1+x^2} - a^2\int_0^{\infty}\frac{\mathrm{d}x}{1+a^2x^2}\right) = \frac{1}{1-a^2}\left(\frac{\pi}{2} - a^2 \cdot \frac{1}{a} \cdot \frac{\pi}{2}\right)
\]
\[
= \frac{1}{1-a^2} \cdot \frac{\pi}{2}(1-a) = \frac{\pi}{2(1+a)}.
\]
積分:$I(a) = \frac{\pi}{2}\ln(1+a)$。
\end{prf}

%=============================================================================
\section{涉及 $\ln(1 + a^2x^2)$ 的積分}
%=============================================================================

\begin{keyintegral}
對於 $a > 0$:
\[
\int_0^{\infty} \frac{\ln(1 + a^2 x^2)}{1 + x^2}\,\mathrm{d}x = \pi\ln(1 + a)
\]
\end{keyintegral}

\begin{prf}
令 $I(a) = \int_0^{\infty} \frac{\ln(1 + a^2x^2)}{1 + x^2}\,\mathrm{d}x$,其中 $I(0) = 0$。

\textbf{微分合理性:}我們應用定理~\ref{thm:leibniz-improper}。偏導數為:
\[
\frac{\partial}{\partial a}\frac{\ln(1+a^2x^2)}{1+x^2} = \frac{2ax^2}{(1+a^2x^2)(1+x^2)}
\]
對於 $a \in [0, A]$,我們有 $\frac{2ax^2}{(1+a^2x^2)(1+x^2)} \leqslant \frac{2Ax^2}{(1+x^2)^2}$,且 $\int_0^{\infty} \frac{x^2}{(1+x^2)^2}\,\mathrm{d}x < \infty$。因此微分是合理的。

微分:
\[
I'(a) = \int_0^{\infty} \frac{2ax^2}{(1 + a^2x^2)(1 + x^2)}\,\mathrm{d}x.
\]
使用部分分式:
\[
\frac{x^2}{(1+a^2x^2)(1+x^2)} = \frac{1}{1-a^2}\left(\frac{1}{1+a^2x^2} - \frac{1}{1+x^2}\right).
\]
故:
\[
I'(a) = \frac{2a}{1-a^2}\left(\frac{\pi}{2a} - \frac{\pi}{2}\right) = \frac{2a}{1-a^2} \cdot \frac{\pi(1-a)}{2a} = \frac{\pi}{1+a}.
\]
因此 $I(a) = \pi\ln(1+a)$。
\end{prf}

%=============================================================================
\section{習題}
%=============================================================================

%\textbf{習題 1.}(量子力學)量子諧振子基態的期望值 $\langle x^4 \rangle$ 需要 $\int_{-\infty}^{\infty} x^4 e^{-\alpha x^2}\,\mathrm{d}x$。透過對 $\int_{-\infty}^{\infty} e^{-ax^2}\,\mathrm{d}x = \sqrt{\pi/a}$ 微分兩次來計算此積分。
%
%\begin{solution}
%令 $I(a) = \int_{-\infty}^{\infty} e^{-ax^2}\,\mathrm{d}x = \sqrt{\frac{\pi}{a}} = \sqrt{\pi}\, a^{-1/2}$。
%
%\textbf{微分合理性:}由定理~\ref{thm:leibniz-improper},我們需要 $\frac{\partial}{\partial a}(e^{-ax^2}) = -x^2 e^{-ax^2}$ 的控制函數。對於 $a \in [a_0, a_1]$($a_0 > 0$),我們有 $|x^2 e^{-ax^2}| \leqslant x^2 e^{-a_0 x^2}$,且 $\int_{-\infty}^{\infty} x^2 e^{-a_0 x^2}\,\mathrm{d}x < \infty$。因此微分是合理的。
%
%對 $a$ 微分:
%\[
%I'(a) = -\int_{-\infty}^{\infty} x^2 e^{-ax^2}\,\mathrm{d}x = \sqrt{\pi} \cdot \left(-\frac{1}{2}\right) a^{-3/2} = -\frac{\sqrt{\pi}}{2a^{3/2}}.
%\]
%
%因此:
%\[
%\int_{-\infty}^{\infty} x^2 e^{-ax^2}\,\mathrm{d}x = \frac{\sqrt{\pi}}{2a^{3/2}}.
%\]
%對於第二次微分,我們再次驗證:$\frac{\partial}{\partial a}(x^2 e^{-ax^2}) = -x^4 e^{-ax^2}$,被 $x^4 e^{-a_0 x^2}$ 控制,後者可積。因此:
%\[
%I''(a) = \int_{-\infty}^{\infty} x^4 e^{-ax^2}\,\mathrm{d}x = -\frac{\mathrm{d}}{\mathrm{d}a}\left(-\frac{\sqrt{\pi}}{2a^{3/2}}\right) = -\frac{\sqrt{\pi}}{2} \cdot \left(-\frac{3}{2}\right) a^{-5/2} = \frac{3\sqrt{\pi}}{4a^{5/2}}.
%\]
%\[
%\boxed{\int_{-\infty}^{\infty} x^4 e^{-\alpha x^2}\,\mathrm{d}x = \frac{3\sqrt{\pi}}{4\alpha^{5/2}}}
%\]
%\end{solution}
%
%\bigskip

%\textbf{習題 2.}(機率論)透過計算 $\int_{-\infty}^{\infty} e^{itx} \cdot \frac{1}{\sqrt{2\pi}}e^{-x^2/2}\,\mathrm{d}x$ 證明標準常態分佈的特徵函數為 $\varphi(t) = e^{-t^2/2}$。
%
%\begin{solution}
%我們需要計算:
%\[
%\varphi(t) = \frac{1}{\sqrt{2\pi}} \int_{-\infty}^{\infty} e^{itx} e^{-x^2/2}\,\mathrm{d}x = \frac{1}{\sqrt{2\pi}} \int_{-\infty}^{\infty} e^{-x^2/2 + itx}\,\mathrm{d}x.
%\]
%
%在指數中配方:
%\[
%-\frac{x^2}{2} + itx = -\frac{1}{2}(x^2 - 2itx) = -\frac{1}{2}(x - it)^2 - \frac{t^2}{2}.
%\]
%
%因此:
%\[
%\varphi(t) = \frac{e^{-t^2/2}}{\sqrt{2\pi}} \int_{-\infty}^{\infty} e^{-(x-it)^2/2}\,\mathrm{d}x.
%\]
%
%積分 $\int_{-\infty}^{\infty} e^{-(x-it)^2/2}\,\mathrm{d}x$ 可以透過移動積分路徑計算(因為被積函數是整函數且快速衰減,這是合理的):
%\[
%\int_{-\infty}^{\infty} e^{-(x-it)^2/2}\,\mathrm{d}x = \int_{-\infty}^{\infty} e^{-u^2/2}\,\mathrm{d}u = \sqrt{2\pi}.
%\]
%
%因此:
%\[
%\boxed{\varphi(t) = e^{-t^2/2}}
%\]
%\end{solution}
%
%\bigskip

\textbf{習題 1.} 證明 $\forall\,a, b > 0$,$\ds \int_0^{\infty} \frac{\sin(ax)\sin(bx)}{x^2}\,\mathrm{d}x = \frac{\pi}{2}\min(a,b)$。

\begin{solution}
使用積化和差公式:
\[
\sin(ax)\sin(bx) = \frac{1}{2}\bigl[\cos((a-b)x) - \cos((a+b)x)\bigr].
\]

因此:
\[
\int_0^{\infty} \frac{\sin(ax)\sin(bx)}{x^2}\,\mathrm{d}x = \frac{1}{2}\int_0^{\infty} \frac{\cos((a-b)x) - \cos((a+b)x)}{x^2}\,\mathrm{d}x.
\]

定義 $J(c) = \int_0^{\infty} \frac{1-\cos(cx)}{x^2}\,\mathrm{d}x$,其中 $J(0) = 0$。

\textbf{微分合理性:}由定理~\ref{thm:leibniz-improper},$\frac{\partial}{\partial c}\frac{1-\cos(cx)}{x^2} = \frac{\sin(cx)}{x}$。收斂性由 Dirichlet 積分理論保證:對於 $c \neq 0$,積分 $\int_0^{\infty} \frac{\sin(cx)}{x}\,\mathrm{d}x$ 條件收斂到 $\frac{\pi}{2}\,\mathrm{sgn}(c)$。

微分:$J'(c) = \int_0^{\infty} \frac{\sin(cx)}{x}\,\mathrm{d}x = \frac{\pi}{2}$(透過代換由 Dirichlet 積分得到)。

積分:對於 $c \geqslant 0$,$J(c) = \frac{\pi c}{2}$。

對於 $c < 0$:由於 $\cos(cx) = \cos(|c|x)$,我們有 $J(c) = J(|c|) = \frac{\pi|c|}{2}$。

因此對所有 $c$,$J(c) = \frac{\pi|c|}{2}$。

現在:
\[
\int_0^{\infty} \frac{\cos((a-b)x) - \cos((a+b)x)}{x^2}\,\mathrm{d}x = J(a+b) - J(|a-b|) = \frac{\pi(a+b)}{2} - \frac{\pi|a-b|}{2}
\]
\[
= \frac{\pi}{2}\bigl((a+b) - |a-b|\bigr) = \frac{\pi}{2} \cdot 2\min(a,b) = \pi\min(a,b)
\]

使用恆等式 $(a+b) - |a-b| = 2\min(a,b)$。

因此:
\[
\int_0^{\infty} \frac{\sin(ax)\sin(bx)}{x^2}\,\mathrm{d}x = \frac{1}{2} \cdot \pi\min(a,b) = \frac{\pi}{2}\min(a,b)
\]

\[
\boxed{\int_0^{\infty} \frac{\sin(ax)\sin(bx)}{x^2}\,\mathrm{d}x = \frac{\pi}{2}\min(a,b)}
\]
\end{solution}

\bigskip

\textbf{習題 2.} $\forall\,a > 0$,求 $\ds\int_0^{\infty} xe^{-ax^2}\sin(bx)\,\mathrm{d}x$。

\begin{solution}
令 $I(b) = \int_0^{\infty} e^{-ax^2}\cos(bx)\,\mathrm{d}x = \frac{1}{2}\sqrt{\frac{\pi}{a}}e^{-b^2/(4a)}$(來自 Gauss Fourier 變換)。

\textbf{微分合理性:}由定理~\ref{thm:leibniz-improper},$\frac{\partial}{\partial b}(e^{-ax^2}\cos(bx)) = -x e^{-ax^2}\sin(bx)$。對所有 $b \in \R$,$|x e^{-ax^2}\sin(bx)| \leqslant x e^{-ax^2}$,且 $\int_0^{\infty} x e^{-ax^2}\,\mathrm{d}x = \frac{1}{2a} < \infty$。因此微分是合理的。

對 $b$ 微分:
\[
I'(b) = -\int_0^{\infty} x e^{-ax^2}\sin(bx)\,\mathrm{d}x.
\]
另外
\[
I'(b) = \frac{1}{2}\sqrt{\frac{\pi}{a}} \cdot e^{-b^2/(4a)} \cdot \left(-\frac{b}{2a}\right) = -\frac{b}{4a}\sqrt{\frac{\pi}{a}}e^{-b^2/(4a)}.
\]
因此
\[
\boxed{\int_0^{\infty} xe^{-ax^2}\sin(bx)\,\mathrm{d}x = \frac{b\sqrt{\pi}}{4a^{3/2}}e^{-b^2/(4a)}}
\]
\end{solution}

\bigskip

\textbf{習題 3.} 證明 $\forall\,a, b, c > 0$,$\ds\int_0^{\infty} \frac{e^{-ax} - e^{-bx}}{x}\sin(cx)\,\mathrm{d}x = \tan^{-1}\frac{c}{a} - \tan^{-1}\frac{c}{b}$

\begin{solution}
由正文,$\forall\alpha > 0$:
\[
J(\alpha) = \int_0^{\infty} e^{-\alpha x}\frac{\sin(cx)}{x}\,\mathrm{d}x = \tan^{-1}\frac{c}{\alpha}.
\]

\textbf{微分合理性:}$J(\alpha) = \tan^{-1}(c/\alpha)$ 的推導使用定理~\ref{thm:leibniz-improper}。設 $f(x, \alpha) = e^{-\alpha x}\frac{\sin(cx)}{x}$,偏導數 $\frac{\partial f}{\partial \alpha} = -e^{-\alpha x}\sin(cx)$ 對於 $\alpha \geqslant \alpha_0 > 0$ 滿足 $|\frac{\partial f}{\partial \alpha}| \leqslant e^{-\alpha_0 x}$,後者可積。

我們要求的積分為:
\[
\int_0^{\infty} \frac{e^{-ax} - e^{-bx}}{x}\sin(cx)\,\mathrm{d}x = \int_0^{\infty} \frac{e^{-ax}\sin(cx)}{x}\,\mathrm{d}x - \int_0^{\infty} \frac{e^{-bx}\sin(cx)}{x}\,\mathrm{d}x
\]
\[
= J(a) - J(b) = \tan^{-1}\frac{c}{a} - \tan^{-1}\frac{c}{b}.
\]

\[
\boxed{\int_0^{\infty} \frac{e^{-ax} - e^{-bx}}{x}\sin(cx)\,\mathrm{d}x = \tan^{-1}\frac{c}{a} - \tan^{-1}\frac{c}{b}}
\]
\end{solution}
\end{document}
