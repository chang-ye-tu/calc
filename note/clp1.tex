\begin{theorem}[Limits and one sided limits]
\begin{align*}
\lim_{x \to a} f(x) = L && \mbox{ if and only if } && \lim_{x \to a^-} f(x) = L \mbox{ and } \lim_{x \to a^+} f(x) = L
\end{align*}
\end{theorem}
\begin{align*}
\lim_{x \to 0} \frac{1}{x^2} &= +\infty & \lim_{x \to 0} -\frac{1}{x^2} &= -\infty
\end{align*}
\section{Calculating Limits with Limit Laws}
\begin{theorem}[Easiest limits]
Let $a, c\in\mathbb{R}$, then
\begin{align*}
\lim_{x \to a} c & = c & \text{ and }&& \lim_{x \to a} x &= a.
\end{align*}
\end{theorem}
\begin{theorem}[Arithmetic of limits]
Let $a$, $c\in\mathbb{R}$, $f(x)$, $g(x)$ be defined for all $x$'s that lie in some interval about $a$ (but $f,g$ need not be defined exactly at $a$), and
\begin{align*}
\lim_{x \to a} f(x)&=F & \lim_{x \to a} g(x) &=G
\end{align*}
exist with $F,G \in \mathbb{R}$. Then
\begin{itemize}
\item $\ds \lim_{x \to a} (f(x) + g(x)) = F + G$ --- limit of the sum is the sum of the limits.
\item $\ds \lim_{x \to a} (f(x) - g(x)) = F - G$ --- limit of the difference is the difference of the limits.
\item $\ds \lim_{x \to a} c f(x) = c F$
\item $\ds \lim_{x \to a} ( f(x) \cdot g(x) ) = F \cdot G$ --- limit of the product is the product of limits.
\item If $G \neq 0$ then $\ds \lim_{x \to a} \frac{f(x)}{g(x)} = \frac{F}{G}$ and, in particular, $\ds \lim_{x\to a}\frac{1}{g(x)}=\frac{1}{G}$.
\end{itemize}
\end{theorem}
\section{Limits at Infinity}
\begin{ex}
\begin{efig}
\begin{center}
\includegraphics[height=4cm]{lim7}
\end{center}
\end{efig}
\end{ex}
\begin{theorem}
Given $c\in\mathbb{R}$,
\begin{align*}
\lim_{x \to \infty} c &= c & \lim_{x \to -\infty} c &= c \\
\lim_{x \to \infty} \frac{1}{x} &= 0 & \lim_{x \to -\infty} \frac{1}{x} &= 0
\end{align*}
\end{theorem}
\begin{theorem}[Arithmetic of limits at infinity]
Given $f(x)$, $g(x)$ with
\begin{align*}
\lim_{x \to \infty} f(x)=F, \qquad\lim_{x \to \infty} g(x) = G,
\end{align*}
($F$, $G\in\mathbb{R}$). Then
\begin{align*}
\lim_{x \to \infty} f(x) \pm g(x) &= F \pm G \\
\lim_{x \to \infty} f(x) g(x) &= F  G \\
\lim_{x \to \infty} \frac{f(x)}{ g(x) } &= \frac{F}{G} & \text{provided $G \neq 0$}
\intertext{and for $p\in\mathbb{R}$}
\lim_{x \to \infty} f(x)^p &= F^p & \text{provided $F^p$ and $f(x)^p$ are defined for all $x$}
\end{align*}
The analogous results hold for limits to $-\infty$.
\end{theorem}
\begin{itemize}
\item $\forall r\in\mathbb{Q}$ with $r > 0$,
\begin{align*}
\lim_{x \to \infty} \frac{1}{x^r} &= 0
\end{align*}
\item Be careful with
\begin{align*}
\lim_{x \to -\infty} \frac{1}{x^r} &= 0
\end{align*}
This is true only if the least denomiator of $r$ is not even!
\item eg. $\ds \lim_{x \to \infty} \frac{1}{x^{1/2}} = 0$, but $\ds \lim_{x \to -\infty} \frac{1}{x^{1/2}}$ does not exist.%, because $x^{1/2}$ is not defined for $x<0$.
\item On the other hand, $x^{4/3}$ is defined for negative values of $x$ and $\ds \lim_{x \to -\infty} \frac{1}{x^{4/3}} = 0$.
\end{itemize}
\begin{theorem}[Arithmetic of infinite limits]
Let $a,c,H \in \mathbb{R}$ and let $f,g,h$ be functions defined in
an interval around $a$ (but they need not be defined at $x=a$), so that
\begin{align*}
  \lim_{x \to a} f(x) &= +\infty &
  \lim_{x \to a} g(x) &= +\infty &
  \lim_{x \to a} h(x) &= H
\end{align*}
Then
\begin{itemize}
\item $\ds \lim_{x \to a} ( f(x) + g(x) ) = +\infty$
\item $\ds \lim_{x \to a} ( f(x) + h(x) ) = +\infty$
\item $\ds \lim_{x \to a} ( f(x) - g(x) )$ undetermined
\item $\ds \lim_{x \to a} ( f(x) - h(x) ) = +\infty$
\item $\ds \lim_{x \to a} c f(x) =
\begin{cases}
+\infty & c>0 \\
-\infty & c<0 \\
0       & c=0
\end{cases}$
\item $\ds \lim_{x \to a} ( f(x) \cdot g(x) ) = +\infty$.
\item $\ds \lim_{x \to a} f(x) h(x) =
\begin{cases}
+\infty & H>0 \\
-\infty & H<0\\
\text{undetermined} & H=0
\end{cases}
$
\item $\ds \lim_{x \to a} \frac{f(x)}{g(x)}$ undetermined
\item $\ds \lim_{x \to a} \frac{f(x)}{h(x)} =
\begin{cases}
+\infty & H>0 \\
-\infty & H<0\\
\text{undetermined} & H=0
\end{cases}$
\item $\ds \lim_{x \to a} \frac{h(x)}{f(x)} = 0$
\item $\ds \lim_{x \to a} f(x)^p =
\begin{cases}
+\infty & p>0 \\
0 & p<0\\
1 & p=0
\end{cases}$
\end{itemize}
\end{theorem}
\begin{itemize}
\item Examples of
\begin{align*}
  \lim_{x \to 0} f(x) = +\infty, \quad \lim_{x \to 0} h(x) = 0 \ie \lim_{x \to 0} f(x) h(x) = \text{undetermined}
\end{align*}
Take $f(x) = \frac{1}{x^4}$, $h(x) = x^2,\,x^3,\,x^4,\,x^5$, then
\begin{align*}
\lim_{x \to 0} f(x) h(x) =
\begin{cases}
\ds\lim_{x \to 0} \frac{1}{x^4} x^2 = \lim_{x \to 0} \frac{1}{x^2} = +\infty \\
\ds\lim_{x \to 0} \frac{1}{x^4} x^3 = \lim_{x \to 0} \frac{1}{x} = \text{DNE}\\
\ds\lim_{x \to 0} \frac{1}{x^4} x^4 = \lim_{x \to 0} 1 = 1                   \\
\ds\lim_{x \to 0} \frac{1}{x^4} x^5 = \lim_{x \to 0} x = 0                   \\
\end{cases}
\end{align*}
\end{itemize}
\begin{itemize}
\item Examples of
\begin{align*}
\lim_{x \to 0} f(x) = +\infty, \quad \lim_{x \to 0} g(x) = +\infty \ie \lim_{x \to 0} \frac{f(x)}{g(x)} = \text{undetermined}
\end{align*}
Take $f(x) = \frac{1}{x^4}$, $g(x) = \frac{1}{x^6},\,\frac{1}{x^4},\,\frac{1}{x^2}$, then
\begin{align*}
\lim_{x \to 0} \frac{f(x)}{g(x)} = \begin{cases}
\ds\lim_{x \to 0}\frac{\frac{1}{x^4}}{\frac{1}{x^6}} = \lim_{x \to 0} x^2 = 0 \\
\ds\lim_{x \to 0}\frac{\frac{1}{x^4}}{\frac{1}{x^4}} = \lim_{x \to 0} 1 = 1 \\
\ds\lim_{x \to 0}\frac{\frac{1}{x^4}}{\frac{1}{x^2}} = \lim_{x \to 0} \frac{1}{x^2} = +\infty
\end{cases}
\end{align*}
\end{itemize}
\begin{ex}
Consider the following 3 functions:
\begin{align*}
f(x)&=x^{-2} & g(x)&=2x^{-2} &h(x)&=x^{-2}-1. \\
\intertext{Their limits as $x \to 0$ are:}
\lim_{x\to0} f(x) &= +\infty &
\lim_{x\to0} g(x) &= +\infty &
\lim_{x\to0} h(x) &= +\infty.
\end{align*}
But
\begin{align*}
\lim_{x\to0} \left( f(x)-g(x) \right) &= \lim_{x\to0} -x^{-2} = -\infty \\
\lim_{x\to0} \left( f(x)-h(x) \right) &= \lim_{x\to0} 1 = 1 \\
\lim_{x\to0} \left( g(x)-h(x) \right) &= \lim_{x\to0} x^{-2}+1 = +\infty
\end{align*}
\end{ex}
\section{Continuity}
\begin{defn}
A function $f(x)$ is continuous at $a$ if
\begin{align*}
  \lim_{x \to a} f(x) &= f(a).
\end{align*}
\begin{itemize}
  \item $f$ is not continuous / discontinuous at $a$
  \item $f$ is continuous: $f$ is continuous at $a$ for all $a \in \mathbb{R}$
  \item $f(x)$ is continuous on $(a,b)$: $f$ is continuous $\forall c$ where $a < c <b$
\end{itemize}
\end{defn}
$f$ is continuous at $a$ then
\begin{itemize}
  \item $f(a)$ exists
  \item $\ds \lim_{x \to a^-}$ exists and is equal to $f(a)$, and
  \item $\ds \lim_{x \to a^+}$ exists and is equal to $f(a)$.
\end{itemize}
\begin{defn}
$f(x)$ is continuous \emph{from the right} at $a$ if
\begin{align*}
\lim_{x\to a^+} f(x) &= f(a).
\end{align*}
$f(x)$ is continuous \emph{from the left} at $a$ if
\begin{align*}
\lim_{x\to a^-} f(x) &= f(a)
\end{align*}
\end{defn}
\begin{defn}
$f(x)$ is continuous on $[a,b]$ when
\begin{itemize}
\item $f(x)$ is continuous on $(a,b)$
\item $f(x)$ is continuous from the right at $a$
\item $f(x)$ is continuous from the left at $b$
\end{itemize}
Note that the last two condition are equivalent to
\begin{align*}
\lim_{x\to a^+} f(x) &= f(a) & \text{ and } && \lim_{x\to b^-} f(x) &= f(b).
\end{align*}
\end{defn}
\begin{itemize}
\item $f(x)$ has a ``jump discontinuity''
\item $g(x)$ has an ``infinite discontinuity''
\item $h(x)$ has a ``removable discontinuity''
\begin{align*}
\text{new function }h(x) &= \begin{cases}
\frac{x^3-x^2}{x-1} & x\neq 1\\
1 & x=1
\end{cases}
\end{align*}
\end{itemize}
\begin{theorem}[Arithmetic of continuity]
Let $a,c \in \mathbb{R}$, $f(x)$ and $g(x)$ be continuous at $a$. Then
\begin{itemize}
\item $f(x) \pm g(x)$
\item $c f(x)$ and $f(x) g(x)$
\item $\frac{f(x)}{g(x)}$ provided $g(a)\neq 0$
\end{itemize}
are continuous at $x=a$.
\end{theorem}
\begin{lemma}
Let $c \in \mathbb{R}$, then
\begin{align*}
f(x) = c,\qquad g(x) = x
\end{align*}
are continuous on $\mathbb{R}$.
\end{lemma}
\begin{defn}
Let $a \in \mathbb{R}$ and %let
$f(x)$ be a function defined everywhere in a neighbourhood of $a$, except possibly at $a$. %We say that
Then $\ds\lim_{x \to a} f(x) = L$ if and only if for every $\epsilon >0$ there exists $\delta>0$ so that
\begin{align*}
  \text{if } 0 < |x-a| < \delta \text{ then } |f(x) - L| <\epsilon.
\end{align*}
In symbols:
\begin{align*}
  \lim_{x \to a} f(x) = L \ie \forall\,\epsilon > 0\;\exists\,\delta > 0\,(0 < |x - a| < \delta \ie |f(x) - L| < \epsilon)
\end{align*}
\end{defn}
\begin{itemize}
\item $\lim_{x \to a} f(x) = L$ via the $\epsilon$-$\delta$ argument: Given an arbitrary $\epsilon > 0$, we should be able to find $\delta > 0$ which makes $|f(x) - L|< \epsilon$ if $0 < |x - a| < \delta$.
\end{itemize}
\begin{itemize}
\item Using $\epsilon$-$\delta$ argument to prove
\begin{align*}
\lim_{x\to 1}x^2 = 1
\end{align*}
\item
Let $\epsilon > 0$ be given. If we can find $\delta$ which makes $|x^2 - 1| < \epsilon$ when $0 < |x - 1| < \delta$ then we're done.
\item Note that $|x^2 - 1| = |(x + 1)(x - 1)|$. If we assume $\delta\leqslant 1$, then from
\begin{align*}
|x - 1| < 1
\end{align*}
we have
\begin{gather*}
-1 < x - 1 < 1 \\
0 < x < 2 \\
1 < x + 1 < 3 \\
|x + 1| < 3
\end{gather*}
Now take $\delta = \min\{1, \frac{\epsilon}{3}\}$; when $0 < |x - 1| < \delta$,
\begin{align*}
|x^2 - 1| = |(x - 1)(x + 1)| < |x + 1||x - 1| < 3\delta \leqslant 3\cdot\frac{\epsilon}{3}= \epsilon,
\end{align*}
as required.
\end{itemize}
\begin{itemize}
\item Using $\epsilon$-$\delta$ argument to prove
\begin{align*}
\lim_{x\to 2}x^2 = 4
\end{align*}
\item
Let $\epsilon > 0$ be given. If we can find $\delta$ which makes $|x^2 - 4| < \epsilon$ when $0 < |x - 2| < \delta$ then we're done.
\item Note that $|x^2 - 4| = |(x + 2)(x - 2)|$. If we assume $\delta\leqslant 1$, then from
\begin{align*}
|x - 2| < 1
\end{align*}
we have
\begin{gather*}
-1 < x - 2 < 1 \\
1 < x < 3 \\
3 < x + 2 < 5 \\
|x + 2| < 5
\end{gather*}
Now take $\delta = \min\{1, \frac{\epsilon}{5}\}$; when $0 < |x - 2| < \delta$,
\begin{align*}
|x^2 - 4| = |(x - 2)(x + 2)| < |x + 2||x - 2| < 5\delta \leqslant 5\cdot\frac{\epsilon}{5}= \epsilon,
\end{align*}
as required.
\end{itemize}
\begin{itemize}
\item Using $\epsilon$-$\delta$ argument to prove
\begin{align*}
\lim_{x\to 3}x^2 = 9
\end{align*}
\item
Let $\epsilon > 0$ be given. If we can find $\delta$ which makes $|x^2 - 9| < \epsilon$ when $0 < |x - 3| < \delta$ then we're done.
\item Note that $|x^2 - 9| = |(x + 3)(x - 3)|$. If we assume $\delta\leqslant 1$, then from
\begin{align*}
|x - 3| < 1
\end{align*}
we have
\begin{gather*}
-1 < x - 3 < 1 \\
2 < x < 4 \\
5 < x + 3 < 7 \\
|x + 3| < 7
\end{gather*}
Now take $\delta = \min\{1, \frac{\epsilon}{7}\}$; when $0 < |x - 3| < \delta$,
\begin{align*}
|x^2 - 9| = |(x - 3)(x + 3)| < |x + 3||x - 3| < 7\delta \leqslant 7\cdot\frac{\epsilon}{7}= \epsilon,
\end{align*}
as required.
\end{itemize}
\begin{itemize}
\item Using $\epsilon$-$\delta$ argument to prove
\begin{align*}
\lim_{x\to a}x^2 = a^2
\end{align*}
\item Alternative way to find $\delta$: Let $\epsilon > 0$ be given.
\begin{itemize}
\item If $a^2\geqslant\epsilon$, we have
\begin{gather*}
-\epsilon  < x^2 - a^2 < \epsilon \\
a^2 - \epsilon < x^2 < a^2 + \epsilon \\
\sqrt{a^2 - \epsilon} < x < \sqrt{a^2 + \epsilon} \\
\sqrt{a^2 - \epsilon} - a < x - a < \sqrt{a^2 + \epsilon} - a \\
\end{gather*}
Take $\delta = \min\{\sqrt{a^2 + \epsilon} - a, |\sqrt{a^2 - \epsilon} - a|\}$ suffices.
\item Else if $a^2 < \epsilon$, we have
\begin{gather*}
-\epsilon  < x^2 - a^2 < \epsilon \\
a^2 - \epsilon < x^2 < a^2 + \epsilon \\
0 < x < \sqrt{a^2 + \epsilon} \\
- a < x - a < \sqrt{a^2 + \epsilon} - a \\
\end{gather*}
Take $\delta = \min\{\sqrt{a^2 + \epsilon} - a, |-a|\}$ suffices.
\end{itemize}
\end{itemize}
\begin{fig}
\begin{center}
\includegraphics[width=\textwidth]{epsdelt1}
\end{center}
\end{fig}
\begin{ex}
Given
\begin{align*}
f(x) &= \begin{cases}
2x & x\neq 3 \\
9 & x=3
\end{cases}
\end{align*}
Prove that
\begin{align*}
\lim_{x\to 3}f(x) = 6
\end{align*}
\begin{proof}
Given $\epsilon > 0$, we want to find $\delta > 0$ such that %and $\delta = \epsilon/2$.
\begin{align*}
0 < |x - 3| < \delta \ie |f(x) - 6| < \epsilon
\end{align*}
Note that $|f(x) - 6| < \epsilon$ for $x\neq 3$ implies $0 < |x - 3| < \frac{\epsilon}{2}$, so take $\delta = \frac{\epsilon}{2}$ suffices.
\end{proof}
\end{ex}
\begin{ex}
\begin{align*}
f(x) &= \begin{cases}
x & x<2 \\
-1 & x=2 \\
x+3 & x>2
\end{cases}
\end{align*}
\begin{efig}
\begin{center}
\includegraphics[width=\textwidth]{epsdelt2}
\end{center}
\end{efig}
\end{ex}
\begin{align*}
\lim_{x \to a} c = c \text{ and } \lim_{x \to a} x = a.
\end{align*}
\begin{proof}%[Proof of Theorem~\ref{thm easy lim}]
\begin{itemize}
\item %Let $\epsilon>0$ and set $f(x)=c$. Choose $\delta=1$, then for any $x$ satisfying $|x-a|<\delta$ (or indeed any real number $x$ at all) we have $|f(x) - c| = 0 < \epsilon$. Hence $\ds \lim_{x \to a} c = c$ as required.
\item %Let $\epsilon>0$ and set $f(x)=x$. Choose $\delta=\epsilon$, then for any $x$ satisfying $|x-a|<\delta$ we have
\end{itemize}
\end{proof}
\stepcounter{section}
\section{Proving the Arithmetic of Limits}
\begin{lemma}[The triangle inequality]
  For any $x,y \in \mathbb{R}$ $\ds|x+y| \leqslant |x| + |y|$
\end{lemma}
\begin{proof}
  \begin{itemize}
    \item (First Proof) Note that $|x| = \max\{x, -x\}$ and $\pm x\leqslant |x|$. From
      \begin{align*}
        x + y \leqslant |x| + y \leqslant |x| + |y|, \\
        -x - y \leqslant |x| - y \leqslant |x| + |y|,
      \end{align*}
      $\ds |x + y|\leqslant |x| + |y|$.
    \item (Second Proof) Note that $\forall x\in\mathbb{R}$, $x^2 = |x|^2$ and $x \leqslant |x|$. From
      \begin{align*}
        |x + y|^2 = (x + y)^2 = x^2 + 2 x y + y^2 \leqslant |x|^2 + 2 |x||y| + |y|^2 = (|x| + |y|)^2,
      \end{align*}
      $|x + y|\leqslant |x| + |y|$.
  \end{itemize}
\end{proof}
From
\begin{align*}
  |x| = |y + (x - y)| &\leqslant |y| + |x - y| \\
  |y| = |x + (y - x)| &\leqslant |x| + |y - x|
\end{align*}
We have $\ds||x| - |y|| \leqslant |x - y|$
\begin{lemma}
  If $\ds \lim_{x \to a} f(x) = F$, $a$, $F\in\mathbb{R}$, then $\exists\,\delta > 0$ such that when $0 < |x - a| < \delta$ we have $|f(x)| < |F| + 1$.
\end{lemma}
\begin{proof}
  Take $\epsilon = 1$, by definition of $\ds \lim_{x \to a} f(x) = F$,
  \begin{align*}
    \exists\,\delta > 0\;(0 < |x - a| < \delta \ie  |f(x) - F| < 1)
  \end{align*}
  From the formula $\ds ||x| - |y|| \leqslant |x - y|$, $\ds ||f(x)| - |F|| \leqslant |f(x) - F| < 1$, hence $\ds-1 < |f(x)| - |F| < 1$, so $\ds |f(x)| < |F| + 1$.
\end{proof}
\begin{lemma}
  If $\ds \lim_{x \to a} f(x) = F$, $a$, $F\in\mathbb{R}$ with $F\ne 0$, then $\exists\,\delta >0$ such that when $0 < |x-a| < \delta$ we have $|f(x)|> \frac{|F|}{2}$.
\end{lemma}
\begin{proof}
  Take $\epsilon = \frac{|F|}{2}$, by definition of $\ds \lim_{x \to a} f(x) = F$, $\ds\exists\,\delta > 0\;(0 < |x - a| < \delta \ie  |f(x) - F| < \frac{|F|}{2})$. From the formula $||x| - |y|| \leqslant |x - y|$, $\ds||f(x)| - |F|| \leqslant |f(x) - F| < \frac{|F|}{2}$, so $\ds-\frac{|F|}{2} < |f(x)| - |F| < \frac{|F|}{2} \ie \frac{|F|}{2} < |f(x)|$.
\end{proof}
\begin{theorem}[Uniqueness of Limit]
  Let $a, F_1, F_2 \in \mathbb{R}$ with
  \begin{align*}
    \lim_{x \to a} f(x) = F_1 \text{ and }\lim_{x \to a} f(x) = F_2
  \end{align*}
  then $F_1 = F_2$.
\end{theorem}
\begin{proof}
  If $F_1\ne F_2$, set $\epsilon = \frac{|F_1 - F_2|}{3}$.
  \begin{itemize}
    \item By $\lim_{x\to a}f(x) = F_1$, $$\exists\,\delta_1 > 0\;(0 < |x - a| < \delta_1 \ie |f(x) - F_1| < \epsilon)$$
    \item By $\lim_{x\to a}f(x) = F_2$, $$\exists\,\delta_2 > 0\;(0 < |x - a| < \delta_2 \ie |f(x) - F_2| < \epsilon)$$
    \item Take $\delta = \min\{\delta_1, \delta_2\}$, then for $ 0 < |x - a| < \delta$,
      \begin{align*}
        3\epsilon = |F_1 - F_2| = |(f(x) - F_2) - (f(x) - F_1)| \leqslant |f(x) - F_2| + |f(x) - F_1| < \epsilon + \epsilon = 2\epsilon
      \end{align*}
      $3 < 2$, a contradiction. Hence $F_1 = F_2$.
  \end{itemize}
\end{proof}
\begin{theorem}[Limit of a Sum]
  If $\ds\lim_{x\to a}f(x) = F$, $\ds\lim_{x\to a}g(x) = G$ then $\ds\lim_{x\to a} f(x) + g(x) = F + G$.
\end{theorem}
\begin{proof}
  Let $\epsilon > 0$ be given.
  \begin{itemize}
    \item By $\ds\lim_{x\to a}f(x) = F$, $$\exists\,\delta_1 > 0\;\Big(0 < |x - a| < \delta_1\ie |f(x) - F| < \frac{\epsilon}{2}\Big)$$
    \item By $\ds\lim_{x\to a}g(x) = G$, $$\exists\,\delta_2 > 0\;\Big(0 < |x - a| < \delta_2\ie |g(x) - G| < \frac{\epsilon}{2}\Big)$$
    \item Take $\delta = \min\{\delta_1, \delta_2\}$, then for $ 0 < |x - a| < \delta$,
      \begin{align*}
        |f(x) + g(x) - (F + G)| &= |(f(x) - F) + (g(x) - G)| \\
        &\leqslant |f(x) - F| + |g(x) - G| \\
        &< \frac{\epsilon}{2} + \frac{\epsilon}{2} = \epsilon.
      \end{align*}
  \end{itemize}
\end{proof}
\begin{theorem}[Limit of a Product]
  If $\ds\lim_{x\to a}f(x) = F$, $\ds\lim_{x\to a}g(x) = G$ then $\ds\lim_{x\to a} f(x)\cdot g(x) = F\cdot G$.
\end{theorem}
\begin{proof}
  Let $\epsilon > 0$ be given.
  \begin{itemize}
    \item By $\ds\lim_{x\to a}f(x) = F$, $$\exists\,\delta_1 > 0\;\Big(0 < |x - a| < \delta_1 \ie |f(x) - F| < \frac{\epsilon}{2(|G| + 1)}\Big)$$
    \item By $\ds\lim_{x\to a}f(x) = F$, $$\exists\,\delta_2 > 0\;\Big(0 < |x - a| < \delta_2 \ie |f(x)| < |F| + 1\Big)$$
    \item By $\ds\lim_{x\to a}g(x) = G$, $$\exists\,\delta_3 > 0\;\Big(0 < |x - a| < \delta_3 \ie |g(x) - G| < \frac{\epsilon}{2(|F| + 1)}\Big)$$
    \item Take $\ds\delta = \min\{\delta_1, \delta_2, \delta_3\}$, then for $ 0 < |x - a| < \delta$,
      \begin{align*}
        |f(x)\cdot g(x) - F\cdot G| &= |f(x)(g(x) - G) + G(f(x) - F)| \\
        &\leqslant |f(x)|\,|g(x) - G| + |G|\,|f(x) - F|\\
        &< (|F| + 1)\cdot\frac{\epsilon}{2(|F| + 1)} + |G|\,\frac{\epsilon}{2(|G| + 1)}\\
        &< \frac{\epsilon}{2} + \frac{\epsilon}{2} = \epsilon.
      \end{align*}
  \end{itemize}
\end{proof}
\begin{theorem}[Limit of a Reciprocal]
  If $\ds\lim_{x\to a}g(x) = G$, $G\ne 0$, then $\ds\lim_{x\to a}\frac{1}{g(x)} = \frac{1}{G}$.
\end{theorem}
\begin{proof}
  Let $\epsilon > 0$ be given. For $G\ne 0$,
  \begin{itemize}
    \item By $\ds\lim_{x\to a}g(x) = G$, $$\exists\,\delta_1 > 0\;\Big(0 < |x - a| < \delta_1 \ie |g(x) - G| < \frac{|G|^2}{2}\epsilon\Big)$$
    \item By $\ds\lim_{x\to a}g(x) = G$, $$\exists\,\delta_2 > 0\;\Big(0 < |x - a| < \delta_2 \ie |g(x)| > \frac{|G|}{2}\Big)$$
    \item Take $\delta = \min\{\delta_1, \delta_2\}$, then for $ 0 < |x - a| < \delta$,
      \begin{align*}
        \bigg|\frac{1}{g(x)} - \frac{1}{G}\bigg| &= \bigg|\frac{g(x) - G}{g(x)G}\bigg| \\
        &\leqslant \frac{\frac{|G|^2}{2}\epsilon}{\frac{|G|}{2}|G|} = \epsilon.
      \end{align*}
    \end{itemize}
\end{proof}
\begin{theorem}[Squeeze theorem] %(or sandwich theorem or pinch theorem)]
  Let $a \in \mathbb{R}$ and let $f, g, h$ be functions s.t.%three functions so that
  \begin{align*}
    f(x) \leqslant g(x) \leqslant h(x)
  \end{align*}
  for all $x$ in an interval around $a$, except possibly at $x=a$. If
  \begin{align*}
    \lim_{x \to a} f(x) &= \lim_{x \to a} h(x) = L
  \end{align*}
  then %it is also the case that
  \begin{align*}
    \lim_{x \to a} g(x) &= L
  \end{align*}
\end{theorem}
\begin{proof}
  Let $\epsilon > 0$ be given.
  \begin{itemize}
    \item By $\ds\lim_{x\to a}f(x) = L$, $$\ds\exists\,\delta_1 > 0\;(0 < |x - a| < \delta_1 \ie |f(x) - L| < \epsilon \ie L - \epsilon < f(x) < L + \epsilon)$$
    \item By $\ds\lim_{x\to a}h(x) = L$, $$\exists\,\delta_2 > 0\;(0 < |x - a| < \delta_2 \ie |h(x) - L| < \epsilon \ie L - \epsilon < h(x) < L + \epsilon)$$
    \item Take $\ds\delta = \min\{\delta_1, \delta_2\}$, then for $ 0 < |x - a| < \delta$,
      \begin{align*}
        L - \epsilon < f(x)\leqslant g(x)\leqslant h(x) < L + \epsilon
      \end{align*}
      Hence
      \begin{align*}
        L - \epsilon < g(x) < L + \epsilon \ie |g(x) - L| < \epsilon
      \end{align*}
    \end{itemize}
\end{proof}
\begin{theorem}
  If $f$ is continuous at $b$ and $\ds \lim_{x \to a} g(x) = b$ then $\ds \lim_{x\to a} f(g(x)) = f(b)$. I.e.
  \begin{align*}
    \lim_{x \to a} f\big(g(x)\big) &= f\big(\lim_{x \to a} g(x)\big)
  \end{align*}
\end{theorem}
\begin{proof}
  Let $\epsilon > 0$ be given. We want to show that
    $$\exists\,\delta > 0\;(0 < |x - a| < \delta \ie |f(g(x)) - f(b)| < \epsilon)$$
  \begin{itemize}
    \item By the continuity of $f$ at $b$, $\lim_{y\to b}f(y) = f(b)$. So
      \begin{align}
        \exists\,\delta_1 > 0\;(0 < |y - b| < \delta_1 \ie |f(y) - f(b)| < \epsilon)
      \end{align}
    \item By $\lim_{x\to a}g(x) = b$, $$\exists\,\delta_2 > 0\;(0 < |x - a| < \delta_2 \ie |g(x) - b| < \delta_1)$$
    \item Take $\delta = \delta_2$ and let $y = g(x)$, then for $ 0 < |x - a| < \delta$ $\ie$ $|g(x) - b| < \delta_1$, by \eqref{eq:cont} $|f(g(x)) - f(b)| < \epsilon$, as required.
  \end{itemize}
\end{proof}

\section{Proofs of the Arithmetic of Derivatives}
Note that, for differentiable $g(x)$
\begin{align*}
  \lim_{h\to 0}g(x + h) = g(x)
\end{align*}
\subsection*{Proof of the Linearity of Differentiation}
\begin{align*}
  \diff{}{x} \big\{\alpha\,f(x) + \beta\,g(x)\big\} &= \lim_{h\to 0}\frac{(\alpha\,f(x+h) + \beta\,g(x+h)) - (\alpha\,f(x) + \beta\,g(x))}{h} \\
  &=\lim_{h\to 0}\frac{\alpha\,(f(x+h) - f(x)) + \beta\,(g(x+h)-g(x))}{h} \\
  &=\lim_{h\to 0}\alpha\,\frac{f(x+h) - f(x)}{h} + \beta\,\frac{g(x+h)-g(x)}{h} \\
  &= \alpha\,f'(x) + \beta\,g'(x)
\end{align*}
\subsection*{Proof of the Product Rule}
\begin{align*}
  \diff{}{x} \big\{f(x)g(x)\big\} &= \lim_{h\to 0}\frac{f(x+h)g(x+h) - f(x)g(x)}{h} \\
  &=\lim_{h\to 0}\frac{f(x+h)g(x+h) - f(x)g(x+h) + f(x)g(x+h) - f(x)g(x)}{h} \\
  &=\lim_{h\to 0}\frac{(f(x+h)- f(x))g(x+h) + f(x)(g(x+h) - g(x))}{h} \\
  &=\lim_{h\to 0}\frac{f(x+h)- f(x)}{h}g(x+h) + f(x)\frac{g(x+h) - g(x)}{h} \\
  &= f'(x)\, g(x) + f(x)\,g'(x)
\end{align*}
\subsection*{Proof of the Quotient Rule}
\begin{align*}
  \diff{}{x} \left\{ \frac{f(x)}{g(x)} \right\} &= \lim_{h\to 0}\frac{\frac{f(x+h)}{g(x+h)} - \frac{f(x)}{g(x)}}{h} =\lim_{h\to 0}\frac{\frac{f(x+h)g(x) - f(x)g(x+h)}{g(x+h)g(x)}}{h} \\
  &=\lim_{h\to 0}\frac{\frac{f(x+h)g(x) - f(x)g(x) + f(x)g(x) - f(x)g(x+h)}{g(x+h)g(x)}}{h} \\
  &=\lim_{h\to 0}\frac{\frac{(f(x+h) - f(x))\,g(x) - f(x)\,(g(x + h) - g(x))}{g(x+h)g(x)}}{h} \\
  &=\lim_{h\to 0}\frac{\frac{f(x+h) - f(x)}{h}g(x) - f(x)\frac{g(x + h) - g(x)}{h}}{g(x+h)g(x)} \\
  &= \frac{f'(x) \, g(x) - f(x) \, g'(x)}{g(x)^2}
\end{align*}
\subsection*{Derivation of the Chain Rule}
\begin{itemize}
  \item Straightforward but flawed:
    \begin{align*}
      \big(f(g(x))\big)' &= \lim_{h\to 0}\frac{f(g(x + h)) - f(g(x))}{h} = \lim_{h\to 0}\frac{f(g(x+h)) - f(g(x))}{g(x + h) - g(x)}\cdot\frac{g(x + h) - g(x)}{h} \\
      & ??\,=\lim_{h\to 0}\frac{f(g(x+h)) - f(g(x))}{g(x + h) - g(x)}\cdot\lim_{h\to 0}\frac{g(x + h) - g(x)}{h} = f'(g(x))\cdot g'(x)
    \end{align*}
  \item For $f$ is differentiable at $u = g(x)$ and $g$ is differentiable at $x$, set
    \begin{align}\label{eq:E(k)}
      E(k) = \frac{f(u + k) - f(u)}{k} - f'(u), \qquad E(0) = 0
    \end{align}
    Then $E(k)$ is continuous at $k = 0$ from
    \begin{align*}
      \lim_{k\to 0}E(k) = f'(u) - f'(u) = 0 = E(0)
    \end{align*}
    From \eqref{eq:E(k)}
    \begin{align*}
      f(u + k) - f(u) = k\big(E(k) + f'(u)\big)
    \end{align*}
    Set $u = g(x)$ and $k = g(x + h) - g(x)$, then $u + k = g(x + h)$; for $g$ is differentiable at $x$,
    \begin{align*}
      \lim_{h\to 0}k = \lim_{h\to 0} g(x + h) - g(x) = 0.
    \end{align*}
    So
    \begin{align*}
      \big(f(g(x))\big)' &= \lim_{h\to 0}\frac{f(g(x + h)) - f(g(x))}{h} = \lim_{h\to 0}\big(f'(g(x)) + E(k)\big)\cdot\frac{g(x + h) - g(x)}{h} \\
      &\,=\lim_{h\to 0}\big(f'(g(x)) + E(k)\big)\cdot\lim_{h\to 0}\frac{g(x + h) - g(x)}{h} = f'(g(x))\cdot g'(x)
    \end{align*}
\end{itemize}
\begin{theorem}[L'H\^opital's Rule]
  \begin{itemize}
    \item $\big(\frac{0}{0}\text{ type}\big)$ If
      \begin{itemize}
        \item $\exists\,\delta > 0$ s.t. $\forall\,x\in(a, a + \delta)$, $f(x)$, $g(x)$ are differentiable and $g(x)\ne 0$.
        \item $\ds\lim_{x\to a+}f(x) = \lim_{x\to a+}g(x) = 0$.
        \item $\ds\lim_{x\to a+}\frac{f'(x)}{g'(x)} = l\in\overline{\mathbb{R}}$.
      \end{itemize}
      Then $\ds\lim_{x\to a+}\frac{f(x)}{g(x)} = l$.
    \item $\big(\frac{\infty}{\infty}\text{ type}\big)$ If
      \begin{itemize}
        \item $\exists\,\delta > 0$ s.t. $\forall\,x\in(a, a + \delta)$, $f(x)$, $g(x)$ are differentiable.
        \item $\ds\lim_{x\to a+}|f(x)| = \lim_{x\to a+}|g(x)| = \infty$.
        \item $\ds\lim_{x\to a+}\frac{f'(x)}{g'(x)} = l\in\overline{\mathbb{R}}$.
      \end{itemize}
      Then $\ds\lim_{x\to a+}\frac{f(x)}{g(x)} = l$.
  \end{itemize}
  Here $a+$ could be $a-$, $a$; $a$ could be $\pm\infty$ (with some minor modification).
\end{theorem}

\subsection{Standard Examples}
\begin{ex}
  $\ds\lim_{x\to 0}\frac{\sin x}{x}$
\end{ex}
\begin{ex}
  $\ds\lim_{x\to 0} \frac{\sin(x)}{\sin(2x)}$
\end{ex}
\begin{ex}
  $\ds\lim_{x\to 0} \frac{\sin(x^2)}{1-\cos x}$
\end{ex}
\begin{warning}[Limit of ratio of derivatives DNE]
  If $\ds\lim_{x\to a}f(x) = 0$ and $\ds\lim_{x\to a}g(x) = 0$ and $\ds\lim_{x\to a}\frac{f'(x)}{g'(x)} = \text{DNE}$ then it is still possible that $\ds\lim_{x\to a}\frac{f(x)}{g(x)} & \text{ exists.}$
Take $\ds a=0,\quad f(x)=x^2\sin\frac{1}{x}, \quad g(x)= x.$ then
  \begin{gather*}
    \lim_{x\to 0} f(x) = 0\quad\text{and}\quad\lim_{x\to 0} g(x) = 0. \\
    f'(x) = 2x\sin\frac{1}{x} -\cos\frac{1}{x}\quad\text{and}\quad g'(x) = 1
  \end{gather*}
  \begin{align*}
    \lim_{x\to 0} \frac{f'(x)}{g'(x)} &= \lim_{x\to 0} \left( 2x\sin\frac{1}{x} -\cos\frac{1}{x}\right) = \text{DNE} \\
    \lim_{x\to 0}\frac{f(x)}{g(x)} &= \lim_{x\to 0}\frac{x^2\sin\frac{1}{x}}{x} =\lim_{x\to 0} x\sin\frac{1}{x}= 0\qquad\text{By squeeze theorem}
  \end{align*}
\end{warning}
\subsection{Variations}
\begin{ex}
  $\ds\lim_{x\to\infty} \frac{\arctan x - \frac{\pi}{2}}{ \nicefrac{1}{x}}$
\end{ex}
\begin{ex}
  $\ds\lim_{x \to \infty} \frac{\ln x}{x}$
\end{ex}
\begin{ex}
  $\ds\lim_{x \to \infty} \frac{5x^2+3x-3}{x^2+1}$
\end{ex}
\begin{ex}
  $\ds\lim_{x\to 0+} \frac{\ln x}{\tan\big(\frac{\pi}{2}-x\big)}$
\end{ex}
\begin{ex}
  $\ds\lim_{x\to\infty} \frac{e^x + e^{-x}}{e^x - e^{-x}}$
\end{ex}
\begin{ex}
  $\ds\lim_{x\to 0}\Big(\underbrace{\frac{1}{x}}_{\to \pm\infty} - \underbrace{\frac{1}{\ln(1+x)}}_{\to\pm\infty}\Big)$
\begin{equation}
\lim_{x\to 0}\Big(\underbrace{\frac{1}{x}}_{\to \pm\infty} - \underbrace{\frac{1}{\ln(1+x)}}_{\to\pm\infty}\Big)
=\lim_{x\to 0}\underbrace{\frac{\ln(1+x)-x}{x\ln(1+x)}}_{\atp{\mathrm{num}\to 0}{\mathrm{den}\to 0}}\tag{E1}
\end{equation}
\begin{align}
\lim_{x\to 0}\underbrace{\frac{\ln(1+x)-x}{x\ln(1+x)}}_{\atp{\mathrm{num}\to 0}{\mathrm{den}\to 0}}
&=\lim_{x\to 0}\frac{\frac{1}{1+x}-1} {\ln(1+x)+\frac{x}{1+x}}
=\lim_{x\to 0}\frac{1-(1+x)} {(1+x)\ln(1+x)+x}\notag\\
&=-\lim_{x\to 0}\underbrace{\frac{x} {(1+x)\ln(1+x)+x}}_{\atp{\mathrm{num}\to 0}{\mathrm{den}\to 1\times 0+0=0}} \tag{E2}
\end{align}
\begin{equation}
-\lim_{x\to 0}\underbrace{\frac{x} {(1+x)\ln(1+x)+x}}_{\atp{\mathrm{num}\to 0}{\mathrm{den}\to 1\times 0+0=0}}
=-\lim_{x\to 0}\underbrace{\frac{1} {\ln(1+x)+\frac{1+x}{1+x}+1}}_{\atp{\mathrm{num}\to 1}{\mathrm{den}\to 0+1+1=2}}
=-\frac{1}{2}
\tag{E3}
\end{equation}
Combining (E1), (E2) and (E3) gives our final answer
  $\ds\lim_{x\to 0}\Big(\frac{1}{x} - \frac{1}{\ln(1+x)}\Big) = -\frac{1}{2}$
\end{ex}
